%!TEX program = pdflatex
%%%%%%%%%%%%%%%%%%%%%%%%%%%%%%%%%%%%%%%%%
% The Legrand Orange Book
% LaTeX Template
% Version 1.4 (12/4/14)
%
% This template has been downloaded from:
% http://www.LaTeXTemplates.com
%
% Original author:
% Mathias Legrand (legrand.mathias@gmail.com)
%
% License:
% CC BY-NC-SA 3.0 (http://creativecommons.org/licenses/by-nc-sa/3.0/)
%
% Compiling this template:
% This template uses biber for its bibliography and makeindex for its index.
% When you first open the template, compile it from the command line with the 
% commands below to make sure your LaTeX distribution is configured correctly:
%
% 1) pdflatex main
% 2) makeindex main.idx -s StyleInd.ist
% 3) biber main
% 4) pdflatex main x 2
%
% After this, when you wish to update the bibliography/index use the appropriate
% command above and make sure to compile with pdflatex several times 
% afterwards to propagate your changes to the document.
%
% This template also uses a number of packages which may need to be
% updated to the newest versions for the template to compile. It is strongly
% recommended you update your LaTeX distribution if you have any
% compilation errors.
%
% Important note:
% Chapter heading images should have a 2:1 width:height ratio,
% e.g. 920px width and 460px height.
%
%%%%%%%%%%%%%%%%%%%%%%%%%%%%%%%%%%%%%%%%%

%----------------------------------------------------------------------------------------
%	PACKAGES AND OTHER DOCUMENT CONFIGURATIONS
%----------------------------------------------------------------------------------------

\documentclass[11pt,fleqn]{book} % Default font size and left-justified equations

\usepackage[top=3cm,bottom=3cm,left=3.2cm,right=3.2cm,headsep=10pt,a4paper]{geometry} % Page margins

\usepackage{xcolor} % Required for specifying colors by name
\definecolor{blue}{rgb}{0.0, 0.18, 0.39}

% Font Settings
\usepackage{avant} % Use the Avantgarde font for headings
%\usepackage{times} % Use the Times font for headings
\usepackage{mathptmx} % Use the Adobe Times Roman as the default text font together with math symbols from the Sym­bol, Chancery and Com­puter Modern fonts

\usepackage{microtype} % Slightly tweak font spacing for aesthetics
\usepackage[utf8]{inputenc} % Required for including letters with accents
\usepackage[T1]{fontenc} % Use 8-bit encoding that has 256 glyphs
\hyphenation{Mi-nis-té-ri-o}


% Index
\usepackage{calc} % For simpler calculation - used for spacing the index letter headings correctly
\usepackage{makeidx} % Required to make an index
\makeindex % Tells LaTeX to create the files required for indexing
\usepackage{verbatim}

\usepackage[colorinlistoftodos,prependcaption,textsize=tiny,linecolor=red,backgroundcolor=red!25,bordercolor=red]{todonotes}
\usepackage{epigraph}
\renewcommand{\textflush}{flushepinormal}
\setlength{\epigraphwidth}{0.8\textwidth}

\usepackage{nameref}
\usepackage{booktabs}
\usepackage{graphicx}
\usepackage{float}
\usepackage{multirow}


% Bibliography
%\usepackage[backend=biber,style=authoryear,autocite=inline, citestyle=authoryear]{biblatex}
\usepackage[style=abnt]{biblatex}
\addbibresource{bibliography.bib} % BibTeX bibliography file
\defbibheading{bibempty}{}
\renewcommand*{\nameyeardelim}{\addcomma\space}

\newcommand{\VER}[1]{\begingroup\color{red}#1\endgroup}

%----------------------------------------------------------------------------------------

\input{structure} % Insert the commands.tex file which contains the majority of the structure behind the template

\begin{document}

\let\cleardoublepage\clearpage

\renewcommand{\chaptername}{Capítulo}
\renewcommand{\figurename}{Fig.}

%----------------------------------------------------------------------------------------
%	TITLE PAGE
%----------------------------------------------------------------------------------------
\begingroup
	\thispagestyle{empty}
	
	\AddToShipoutPicture*{\put(0,0){\includegraphics[scale=1]{capa}}} % Image background
	
	\AddToShipoutPicture*{\put(116,650){\includegraphics[scale=.75]{brasao.png}}} % Image background
	
	\AddToShipoutPicture*{\put(244,200){\includegraphics[scale=0.2]{ifgvertical}}} % Image background
	
	\vspace*{4.5cm}
	
	\centering
	\par
	{\Huge Projeto Pedagógico}\vspace*{1.5cm}
	\par
	\fontsize{40}{40}
	\selectfont
	Tecnologia em Análise e Desenvolvimento de Sistemas
	\vspace*{10cm}
	\par
	{\Huge 2020}
	\par
\endgroup
\pagebreak

%----------------------------------------------------------------------------------------
%	PEOPLE PAGE
%----------------------------------------------------------------------------------------
\chapterimage{banner3} % Chapter heading image
\begin{center}
	\par
	{\large PRESIDENTE DA REPÚBLICA \\ Jair Messias Bolsonaro}\vspace*{1cm}
	\par
	{\large MINISTRO DA EDUCAÇÃO \\ Nome do Ministro}\vspace*{1cm}
	\par
	{\large SECRETÁRIO DE EDUCAÇÃO PROFISSIONAL E TECNOLÓGICA \\ Alexandro Ferreira de Souza}\vspace*{1cm}
	\par
	{\large REITOR DO INSTITUTO FEDERAL DE GOIÁS \\ Jerônimo Rodrigues da Silva}\vspace*{1cm}
	\par
	{\large PRÓ-REITOR DE PESQUISA E PÓS-GRADUAÇÃO \\ Ruberley Rodrigues de Souza}\vspace*{1cm}
	\par
	{\large DIRETORIA DE PÓS-GRADUAÇÃO \\ Clarinda Aparecida da Silva}\vspace*{1cm}
	\par
	{\large COORDENADOR DO CURSO \\ Nome do Coordenador}\vspace*{1cm}
\end{center}

\chapterimage{banner3} % Chapter heading image
\renewcommand\contentsname{Sumário}
\tableofcontents

%----------------------------------------------------------------------------------------
%	CHAPTER
%----------------------------------------------------------------------------------------
\chapterimage{01.jpg} % Chapter heading image
\chapter{Apresentação}\label{apresentacao}
\vspace{6em}
\begin{flushright}
	\textit{\textcolor{white}{Um bonita citação...}}
\end{flushright}
\vspace{12em}

\todo[inline]{NNNNNNNNNNNNN}~\parencite{Resolucao3De2002}


%-----------------------------------------------
\newpage  
%------------------------------------------------
\section{Identificação do Curso}
\begin{itemize}
	\item \textbf{Instituição Ofertante:} Instituto Federal de Educação, Ciência e Tecnologia de Goiás
	\item \textbf{Nome do curso:} Tecnologia em Análise e Desenvolvimento de Sistemas
	\item \textbf{Carga Horária do Curso:} 2425 horas
	\item \textbf{Forma de oferta:} Presencial
	\item \textbf{Duração:} 2,5 anos
	\item \textbf{Número de Vagas:} 30 vagas anuais
	\item \textbf{Local de Oferta:} Instituto Federal de Goiás - Câmpus Formosa
	\item \textbf{Reitor:} NNNNNNNNNNNNN
	\item \textbf{Pró-Reitora de Ensino:} NNNNNNNNNNNNN
	\item \textbf{Pró-Reitor de Pesquisa, Pós-Graduação e Inovação:} NNNNNNNNNNNNN
	\item \textbf{Diretoria de Pós-Graduação:} NNNNNNNNNNNNN
\end{itemize}

\section{Elaboração do Projeto de Curso}
\begin{itemize}[label=\bfseries]
	\item ...
	\item Waldeyr Mendes Cordeiro da Silva
\end{itemize}

%----------------------------------------------------------------------------------------
%	CHAPTER
%----------------------------------------------------------------------------------------
\chapterimage{02.jpg} % Chapter heading image
\chapter{Introdução}\label{introducao}
\vspace{6em}
\begin{flushright}
	\textit{\textcolor{white}{Um bonita citação...}}
\end{flushright}
\vspace{12em}

\todo[inline]{NNNNNNNNNNNNN}

\section{Justificativa}

\todo[inline]{NNNNNNNNNNNNN}

\section{Público Alvo}

\todo[inline]{NNNNNNNNNNNNN}

\section{Objetivos}\label{objetivos}

\subsection{Objetivo Geral}

\todo[inline]{NNNNNNNNNNNNN}

\subsection{Objetivos Específicos}

\begin{itemize}
\item NNNNNNNNNNNNN 
\end{itemize}


\section{Perfil do Egresso}
O Tecnólogo em Análise e Desenvolvimento de Sistemas:
\begin{enumerate}
	\item Analisa, projeta, desenvolve, testa, implanta e mantém sistemas computacionais de informação. 
	\item Avalia, seleciona, especifica e utiliza metodologias, tecnologias e ferramentas da Engenharia de Software, linguagens de programação e bancos de dados. 
	\item Coordena equipes de produção de softwares.
	\item Vistoria, realiza perícia, avalia, emite laudo e parecer técnico em sua área de formação.
	\item Compreende e domina os fundamentos científico-tecnológicos do seu trabalho.
	\item Respeita as diversidades e os direitos humanos, tem atitude ética e responsabilidade social e ambiental no trabalho e fora dele.
	\item Compreende a importância e impacto do seu trabalho.
	\item Usa a linguagem para a cidadania e profissão.
	\item Trabalha em equipe e conhece planejamento estratégico, iniciativa e liderança.
\item 	Atualiza-se mantendo-se criativo e responsável.
\end{enumerate}



%----------------------------------------------------------------------------------------
%	CHAPTER
%----------------------------------------------------------------------------------------
\chapterimage{03.jpg} % Chapter heading image
\chapter{Organização do Curso}\label{organizacao}
\vspace{6em}
\begin{flushright}
	\textit{\textcolor{white}{Um bonita citação...}}
\end{flushright}
\vspace{12em}

\section{Requisitos para Acesso ao Curso}

\todo[inline]{NNNNNNNNNNNNN}

\section{Forma de Oferta}\label{carga}


\section{Matriz Curricular}\label{matriz}

\todo[inline]{NNNNNNNNNNNNN}
%\caption{Matriz Curricular do Curso Superior em Tecnologia em Análise e Desenvolvimento de Sistemas.}


% Please add the following required packages to your document preamble:
% \usepackage{booktabs}
% \usepackage{graphicx}
\begin{table}[]
	\centering
	\caption{Matriz Curricular do Curso Superior em Tecnologia em Análise e Desenvolvimento de Sistemas.}
	\label{tab:matriz}
	\resizebox{\textwidth}{!}{%
		\begin{tabular}{@{}|l|c|c|c|c|c|c|@{}}
			\toprule
			\textbf{Disciplinas}                                          & \textbf{1º semestre} & \textbf{2º semestre} & \textbf{3º semestre} & \textbf{4º semestre} & \textbf{5º semestre} & \textbf{CH} \\ \midrule
			Algoritmos                                                    & 54                   &                      &                      &                      &                      &             \\ \midrule
			Fundamentos da Computação                                     & 54                   &                      &                      &                      &                      &             \\ \midrule
			Engenharia de Software                                        & 54                   &                      &                      &                      &                      &             \\ \midrule
			Matemática                                                    & 54                   &                      &                      &                      &                      &             \\ \midrule
			Inglês Acadêmico                                              & 54                   &                      &                      &                      &                      &             \\ \midrule
			Leitura e Produção de Textos                                  & 54                   &                      &                      &                      &                      &             \\ \midrule
			Sociologia do Trabalho                                        & 27                   &                      &                      &                      &                      &             \\ \midrule
			Educação Ambiental                                            & 27                   &                      &                      &                      &                      &             \\ \midrule
			Legislação Aplicada                                           & 27                   &                      &                      &                      &                      &             \\ \midrule
			& 405                  &                      &                      &                      &                      & 405         \\ \midrule
			Sistemas Operacionais                                         &                      & 54                   &                      &                      &                      &             \\ \midrule
			Bancos de Dados                                               &                      & 54                   &                      &                      &                      &             \\ \midrule
			Estruturas de Dados                                           &                      & 54                   &                      &                      &                      &             \\ \midrule
			Programação para Web I                                        &                      & 54                   &                      &                      &                      &             \\ \midrule
			Componente Curricular Técnico I                               &                      & 81                   &                      &                      &                      &             \\ \midrule
			Componente Curricular Politécnico I                           &                      & 54                   &                      &                      &                      &             \\ \midrule
			Probabilidade e Estatística                                   &                      & 54                   &                      &                      &                      &             \\ \midrule
			&                      & 405                  &                      &                      &                      & 405         \\ \midrule
			Análise e Desenvolvimento de Sistemas I                       &                      &                      & 81                   &                      &                      &             \\ \midrule
			Redes de Computadores                                         &                      &                      & 54                   &                      &                      &             \\ \midrule
			Programação para Web II                                       &                      &                      & 54                   &                      &                      &             \\ \midrule
			Interação Humano Computador                                   &                      &                      & 27                   &                      &                      &             \\ \midrule
			Administração de Serviços para Internet                       &                      &                      & 54                   &                      &                      &             \\ \midrule
			Componente Curricular Politécnico II                          &                      &                      & 54                   &                      &                      &             \\ \midrule
			Componente Curricular Politécnico III                         &                      &                      & 54                   &                      &                      &             \\ \midrule
			Componente Curricular de Humanidades I                        &                      &                      & 27                   &                      &                      &             \\ \midrule
			&                      &                      & 405                  &                      &                      & 405         \\ \midrule
			Análise e desenvolvimento de sistemas II                      &                      &                      &                      & 81                   &                      &             \\ \midrule
			Arquitetura e Projeto de Software                             &                      &                      &                      & 81                   &                      &             \\ \midrule
			Gerência e Governança em Projetos de Tecnologia da Informação &                      &                      &                      & 54                   &                      &             \\ \midrule
			Componente Curricular Técnico II                              &                      &                      &                      & 81                   &                      &             \\ \midrule
			Metodologia e Iniciação Científica                            &                      &                      &                      & 54                   &                      &             \\ \midrule
			Componente Curricular Politécnico IV                          &                      &                      &                      & 54                   &                      &             \\ \midrule
			TCC I**                                                       &                      &                      &                      & 54                   &                      &             \\ \midrule
			&                      &                      &                      & 405                  &                      & 405         \\ \midrule
			Análise e desenvolvimento de sistemas III                     &                      &                      &                      &                      & 81                   &             \\ \midrule
			Segurança da Informação                                       &                      &                      &                      &                      & 54                   &             \\ \midrule
			Inteligência Artificial                                       &                      &                      &                      &                      & 54                   &             \\ \midrule
			Componente Curricular Técnico III                             &                      &                      &                      &                      & 81                   &             \\ \midrule
			Componente Curricular Politécnico V                           &                      &                      &                      &                      & 54                   &             \\ \midrule
			Componente Curricular Politécnico VI                          &                      &                      &                      &                      & 54                   &             \\ \midrule
			Componente Curricular Eletivo Humanas*                        &                      &                      &                      &                      & 27                   &             \\ \midrule
			TCC II**                                                      &                      &                      &                      &                      & 54                   &             \\ \midrule
			&                      &                      &                      &                      & 405                  & 405         \\ \midrule
			Carga Horária em Disciplinas                                  &                      &                      &                      &                      &                      & 2025        \\ \midrule
			Carga Horária para Horas Complementares                       &                      &                      &                      &                      &                      & 175         \\ \midrule
			TOTAL                                                         &                      &                      &                      &                      &                      & 2200        \\ \midrule
			Carga Horária para TCC                                        &                      &                      &                      &                      &                      & 108         \\ \bottomrule
		\end{tabular}%
	}
\end{table}


\section{Metodologia de Ensino-Aprendizagem}\label{metodologia}


\subsection{Trabalhos Discentes}\label{trabdiscentes}

\subsubsection{Disciplinas}\label{dsiciplinas}

\todo[inline]{NNNNNNNNNNNNN}

\subsubsection{Atividades Não Presenciais}

Definição. Arcabouço jurídico.


\subsubsection{Avaliação}

\todo[inline]{NNNNNNNNNNNNN}

\section{Certificação}

O Certificado será emitido pelo Instituto Federal de Educação, Ciência e Tecnologia de Goiás, nos termos da Resolução CNE/CES n.º 1, de 8 de junho de 2007.	
Para obter o Certificado de graduado em ``Tecnologia em Análise e Desenvolvimento de Sistemas'', o discente deverá satisfazer as seguintes exigências:
\begin{itemize}
	\item Ser aprovado em todas as disciplinas do curso com nota mínima igual a 6,0 (seis) e freqüência igual ou superior a 75\% da carga horária da disciplina;
	\item Ser aprovado em defesa pública do trabalho de conclusão de curso perante uma banca composta por, no mínimo, três professores (orientador, mais dois professores convidados);
	\item Possuir pelo menos um certificado que comprove a apresentação (pôster ou oral) de resultados relacionados ao trabalho de conclusão de curso em evento científico (congressos, seminários, simpósios);
	\item Quitação de todas as obrigações junto ao Câmpus Formosa do Instituto Federal de Educação, Ciência e Tecnologia de Goiás;
\end{itemize}

%----------------------------------------------------------------------------------------
%	CHAPTER
%----------------------------------------------------------------------------------------
%------------------------------------------------
\chapterimage{04.jpg} % Chapter heading image
\chapter{Ementas}\label{ementas}
\vspace{6em}
\begin{flushright}
	\textit{\textcolor{white}{Um bonita citação...}}
\end{flushright}
\vspace{12em}

\todo[inline]{NNNNNNNNNNNNN}

\newpage
\section{Algoritmos}\label{disc:algoritmos}

\begin{itemize}
	\item \textbf{Carga horária (hora/aula):} 54
	\item \textbf{Docente Responsável:}~\nameref{WaldeyrMendes}
	\item \textbf{Ementa:} 
	Conceitos de algoritmos;
	Conceitos de linguagens de programação;
	Constantes e variáveis;
	Tipos de dados;
	Operadores e expressões aritméticas, lógicas e literais; 
	Comandos básicos;
	Estruturas condicionais e de repetição;
	Vetores e matrizes;
	Estruturas de dados básicas;
	Modularização;
	Recursividade;
	Algoritmos e meio ambiente;
	\item \textbf{Bibliografia básica}
	\begin{enumerate}
		\item CORMEN T. H., LEISERSON C. E., RIVEST R. L.,STEIN C..; Algoritmos: Teoria e prática. Editora Câmpus, Tradução da 3a ed. americana, São Paulo-SP, 2012.
		\item 
	\end{enumerate}
	\item \textbf{Bibliografia complementar}
	\begin{enumerate}
		\item NNNNNNNNNNNNN
	\end{enumerate}	
\end{itemize}



%----------------------------------------------------------------------------------------
%	CHAPTER X
%----------------------------------------------------------------------------------------
%------------------------------------------------
\chapterimage{05.jpg} % Chapter heading image
\chapter{Estrutura Física}\label{estrutura}
\vspace{6em}
\begin{flushright}
	\textit{\textcolor{white}{Um bonita citação...}}
\end{flushright}
\vspace{12em}

\section{Laboratórios de Informática}
Dois laboratórios de informática com capacidade para até 30 estudantes, com acesso à Internet, computadores com sistema operacional Linux, softwares diversos.


\section{Laboratório de Fisiologia Vegetal}
Equipado com: estufa de secagem, 3 estereoscópios, 3 microscópicos, geladeira, bancadas, 28 cadeiras, quadro e acervo didático (frutos, sementes e folhas herborizadas). 

\section{Laboratório de Bioqímica}
Equipado com: Balanças analítica e semi-analítica, chapas de aquecimento (com agitação magnética), analisador bioquímico, capela de fluxo laminar, agitadores de tubo de ensaio, banho-maria, bomba de vácuo, autoclave, estufas, destilador e deionizador de água e outros.

\section{Laboratório de Anatomia e Zoologia}
Equipado com: Bonecos anatômicos (de abdome) completos, conjuntos anatômicos artificiais de sistemas reprodutores femininos e masculinos, esqueletos completos (artificiais), amostras de animais (do cerrado e de outros biomas) conservados em frascos para visualização, animais empalhados, algumas peças anatômicas naturais de animais, lupas, microscópios, material para coleta de animais e saídas de campo, materiais e reagentes para o empalhamento de animais e outros.

\section{Laboratório de Microscopia e Microbiologia}
Equipado com:  25 microscópios e material para produção de lâminas (lâminas de corte, lâminas e lamínulas de vidro, corantes, fixadores, etc); Lupas, coleções de laminários e outros.

\section{Laboratório de Físico-Química}
Equipado com: pHmetros, destilador, capela de exaustão, estufa, banho-maria, balanças analítica e semi-analítica, deionizador, reator, aparelho de ponto de fusão,  e outros.

\section{Laboratório de Águas Residuais}
Equipado com: Condutivímetros, muflas, banho - maria, bomba de vácuo, analisador de oxigênio dissolvido, turbdímetro, estufa, balança, phmetro, destilador e outros.

\section{Laboratório de Ensino}
Espaço acadêmico voltado ao desenvolvimento e disseminação de tecnologias educacionais voltadas ao ensino de Ciências e Biologia.  Equipado com: acervo didático constituído por jogos, maquetes e representações físicas de organismos e processos biológicos.

\section{Laboratório de Física e Matemática}
O Laboratório de Física possui diversos equipamentos que contribui para o desenvolvimento das atividades experimentais nas áreas de mecânica, óptica, hidrostática, termologia e eletricidade.

\section{Biblioteca}
Biblioteca equipada com áreas de estudo individual e coletivo, 6 computadores com acesso ao portal de periódicos e acervo cerca de 7 mil exemplares, entre livros, livros em braile, cds, dvds e mapas;

\section{Teatro}
Teatro equipado com som e iluminação específica e acomodações para 320 pessoas sentadas;

\section{Outros Espaços}
3 salas para estudos coletivos e reuniões equipadas com mesas, cadeiras e televisor.


%----------------------------------------------------------------------------------------
%	CHAPTER X
%----------------------------------------------------------------------------------------
%------------------------------------------------
\chapterimage{06.jpg} % Chapter heading image
\chapter{Corpo Docente}\label{docentes}
\vspace{6em}
\begin{flushright}
	\textit{\textcolor{white}{Um bonita citação...}}
\end{flushright}
\vspace{12em}

\section{Waldeyr Mendes Cordeiro da Silva}\label{WaldeyrMendes}
\begin{itemize}
	\item Formação Básica: Sistemas de Informação e Ciências Biológicas 
	\item Titulação Máxima: Doutor em Ciências Biológicas (Bioinformática)
	\item Regime de Trabalho: Dedicação Exclusiva
	\item \includegraphics[scale=.03]{Pictures/lattes}~\href{http://lattes.cnpq.br/2391349697609978}{Lattes: http://lattes.cnpq.br/2391349697609978}
	\item \includegraphics[scale=.15]{Pictures/orcid}~\href{https://orcid.org/0000-0002-8660-6331}{ORCID: https://orcid.org/0000-0002-8660-6331}
\end{itemize}

% ----------------------------------------------------------------------------------------
% 	BIBLIOGRAPHY
% ----------------------------------------------------------------------------------------
%----------------------------------------------------------------------------------------
%	CHAPTER X
%----------------------------------------------------------------------------------------
%------------------------------------------------
\chapterimage{07.jpg} % Chapter heading image
%\chapter*{Referências Bibliográficas}
%\bibliography{bibliography}
%\renewcommand\bibname{Referências Bibliográficas}

\chapter*{Referências Bibliográficas}\label{referencias}
\vspace{6em}
\begin{flushright}
	\textit{\textcolor{white}{Um bonita citação...}}
\end{flushright}
\vspace{12em}
%\addcontentsline{toc}{chapter}{\textcolor{verde}{Bibliography}}
%\section{Books}
%\addcontentsline{toc}{section}{Books}
%\printbibliography[heading=bibempty,type=book]
%\section{Articles}
%\addcontentsline{toc}{section}{Articles}
%\printbibliography[heading=bibempty,type=article]
\printbibliography[heading=bibempty]


%----------------------------------------------------------------------------------------

%----------------------------------------------------------------------------------------
%	INDEX
%----------------------------------------------------------------------------------------
%
%\cleardoublepage
%\phantomsection
%\setlength{\columnsep}{0.75cm}
%\addcontentsline{toc}{chapter}{\textcolor{verde}{Index}}
%\printindex


%----------------------------------------------------------------------------------------
%	CHAPTER X
%----------------------------------------------------------------------------------------
%------------------------------------------------
\chapterimage{08.jpg} % Chapter heading image
\chapter{Anexos}\label{conheca}
\vspace{6em}
\begin{flushright}
	\textit{\textcolor{white}{}}
\end{flushright}
\vspace{12em}

\newpage
\section{Compatibilidade com a Matriz Curricular Anterior}
\indent

Com a atualização da Matriz Curricular do Curso Superior em Tecnologia em Análise e Desenvolvimento de Sistemas, estudantes matriculados na matriz anterior poderão migrar para a nova matriz respeitadas as compatibilidades definidas na Tabela~\ref{tab:compatibilidade}.
\begin{table}[H]
	\centering
	\caption{Tabela de compatibilidade de disciplinas entre a matriz curricular antiga e a atualizada.}
	\label{tab:compatibilidade}
	\resizebox{\textwidth}{!}{%
		\begin{tabular}{|l|l|}
			\hline
			\textbf{Disciplinas da Matriz Atuaizada}  & \textbf{Disciplinas da Matriz antiga}                                                            \\ \hline
			Algoritmos                                & Algoritmos                                                                                       \\ \hline
			Fundamentos da Computação                 & Fundamentos da Computação                                                                        \\ \hline
			Engenharia de Software                    & Engenharia de Software                                                                           \\ \hline
			Matemática                                & Matemática Elementar                                                                             \\ \hline
			Inglês Acadêmico                          & -                                                                                                \\ \hline
			Leitura e Produção de Textos              & -                                                                                                \\ \hline
			Sociologia do Trabalho                    & Sociologia do Trabalho                                                                           \\ \hline
			Educação Ambiental                        & Educação Ambiental                                                                               \\ \hline
			Legislação Aplicada                       & Ética e Legislação Aplicada à Informática                                                        \\ \hline
			Sistemas Operacionais                     & Sistemas Operacionais                                                                            \\ \hline
			Bancos de Dados                           & Banco de Dados I                                                                                 \\ \hline
			Estruturas de Dados                       & Estrutura de Dados I                                                                             \\ \hline
			Programação para Web I                    & Programação para Web I                                                                           \\ \hline
			Componente Curricular Eletivo Técnico     & Métodos e Técnicas de Programação                                                                \\ \hline
			Componente Curricular Eletivo Politécnico & Lógica Computacional                                                                             \\ \hline
			Probabilidade e Estatística               & Introdução à Probabilidade e Estatística + Cálculo Diferencial e Integral                        \\ \hline
			Análise e desenvolvimento de sistemas I   & Programação Orientada a Objetos                                                                  \\ \hline
			Redes de Computadores                     & Redes de Computadores                                                                            \\ \hline
			Programação para Web II                   & Programação para Web II                                                                          \\ \hline
			Gerência de Projetos                      & Gerência de Projetos                                                                             \\ \hline
			Administração de Serviços para Internet   & Administração de Serviços para Internet                                                          \\ \hline
			Componente Curricular Eletivo Politécnico & Cálculo Diferencial e Integral                                                                   \\ \hline
			Componente Curricular Eletivo Politécnico & Banco de Dados II                                                                                \\ \hline
			Componente Curricular Eletivo Humanas     & Optativa 1                                                                                       \\ \hline
			Análise e desenvolvimento de sistemas II  & Arquitetura e Projeto de Software                                                                \\ \hline
			Programação para Dispositivos Móveis      & -                                                                                                \\ \hline
			Gerência e Governança em Tecnologia da Informação                & -                                                                                                \\ \hline
			Componente Curricular Eletivo Técnico     & Estrutura de Dados II                                                                            \\ \hline
			Metodologia e Iniciação Científica        & Metodologia da Pesquisa Científica + Tópicos Avançados 1                                         \\ \hline
			Componente Curricular Eletivo Politécnico & Engenharia de Requisitos                                                                         \\ \hline
			TCC I                                     & TCC I                                                                                            \\ \hline
			Análise e desenvolvimento de sistemas III & Análise Orientada a Objetos + Fundamentos de Sistemas de Informação + Interface Homem Computador \\ \hline
			Segurança da Informação                   & Segurança e Auditoria de Sistemas                                                                \\ \hline
			Inteligência Artificial                   & -                                                                                                \\ \hline
			Componente Curricular Eletivo Técnico     & Sistemas Distribuídos                                                                            \\ \hline
			Componente Curricular Eletivo Politécnico & Computação Gráfica e Sistemas Multimídia                                                         \\ \hline
			Componente Curricular Eletivo Politécnico & Qualidade de Software                                                                            \\ \hline
			Componente Curricular Eletivo Humanas     & Optativa 2                                                                                       \\ \hline
			TCC II                                    & TCC II                                                                                           \\ \hline
		\end{tabular}%
	}
\end{table}






\end{document}