%!TEX program = pdflatex
%%%%%%%%%%%%%%%%%%%%%%%%%%%%%%%%%%%%%%%%%
% The Legrand Orange Book
% LaTeX Template
% Version 1.4 (12/4/14)
%
% This template has been downloaded from:
% http://www.LaTeXTemplates.com
%
% Original author:
% Mathias Legrand (legrand.mathias@gmail.com)
%
% License:
% CC BY-NC-SA 3.0 (http://creativecommons.org/licenses/by-nc-sa/3.0/)
%
% Compiling this template:
% This template uses biber for its bibliography and makeindex for its index.
% When you first open the template, compile it from the command line with the 
% commands below to make sure your LaTeX distribution is configured correctly:
%
% 1) pdflatex main
% 2) makeindex main.idx -s StyleInd.ist
% 3) biber main
% 4) pdflatex main x 2
%
% After this, when you wish to update the bibliography/index use the appropriate
% command above and make sure to compile with pdflatex several times 
% afterwards to propagate your changes to the document.
%
% This template also uses a number of packages which may need to be
% updated to the newest versions for the template to compile. It is strongly
% recommended you update your LaTeX distribution if you have any
% compilation errors.
%
% Important note:
% Chapter heading images should have a 2:1 width:height ratio,
% e.g. 920px width and 460px height.
%
%%%%%%%%%%%%%%%%%%%%%%%%%%%%%%%%%%%%%%%%%

%----------------------------------------------------------------------------------------
%	PACKAGES AND OTHER DOCUMENT CONFIGURATIONS
%----------------------------------------------------------------------------------------

\documentclass[11pt,fleqn]{book} % Default font size and left-justified equations

\usepackage[top=3cm,bottom=3cm,left=3.2cm,right=3.2cm,headsep=10pt,a4paper]{geometry} % Page margins

\usepackage{xcolor} % Required for specifying colors by name
\definecolor{blue}{rgb}{0.0, 0.18, 0.39}

% Font Settings
\usepackage{avant} % Use the Avantgarde font for headings
%\usepackage{times} % Use the Times font for headings
\usepackage{mathptmx} % Use the Adobe Times Roman as the default text font together with math symbols from the Sym­bol, Chancery and Com­puter Modern fonts

\usepackage{microtype} % Slightly tweak font spacing for aesthetics
\usepackage[utf8]{inputenc} % Required for including letters with accents
\usepackage[T1]{fontenc} % Use 8-bit encoding that has 256 glyphs
\hyphenation{Mi-nis-té-ri-o}


% Index
\usepackage{calc} % For simpler calculation - used for spacing the index letter headings correctly
\usepackage{makeidx} % Required to make an index
\makeindex % Tells LaTeX to create the files required for indexing
\usepackage{verbatim}

\usepackage[colorinlistoftodos,prependcaption,textsize=tiny,linecolor=red,backgroundcolor=red!25,bordercolor=red]{todonotes}
\usepackage{epigraph}
\renewcommand{\textflush}{flushepinormal}
\setlength{\epigraphwidth}{0.8\textwidth}

\usepackage{nameref}
\usepackage{booktabs}
\usepackage{graphicx}
\usepackage{float}
\usepackage{multirow}


% Bibliography
%\usepackage[backend=biber,style=authoryear,autocite=inline, citestyle=authoryear]{biblatex}
\usepackage[style=abnt]{biblatex}
\addbibresource{bibliography.bib} % BibTeX bibliography file
\defbibheading{bibempty}{}
\renewcommand*{\nameyeardelim}{\addcomma\space}

\newcommand{\VER}[1]{\begingroup\color{red}#1\endgroup}

%----------------------------------------------------------------------------------------

\input{structure} % Insert the commands.tex file which contains the majority of the structure behind the template

\begin{document}

\let\cleardoublepage\clearpage

\renewcommand{\chaptername}{Capítulo}
\renewcommand{\figurename}{Fig.}

%----------------------------------------------------------------------------------------
%	TITLE PAGE
%----------------------------------------------------------------------------------------
\begingroup
	\thispagestyle{empty}
	
	\AddToShipoutPicture*{\put(0,0){\includegraphics[scale=1]{capa}}} % Image background
	
	\AddToShipoutPicture*{\put(116,650){\includegraphics[scale=.75]{brasao.png}}} % Image background
	
	\AddToShipoutPicture*{\put(244,200){\includegraphics[scale=0.2]{ifgvertical}}} % Image background
	
	\vspace*{4.5cm}
	
	\centering
	\par
	{\Huge Projeto Pedagógico}\vspace*{1.5cm}
	\par
	\fontsize{40}{40}
	\selectfont
	Tecnologia em Análise e Desenvolvimento de Sistemas
	\vspace*{10cm}
	\par
	{\Huge 2020}
	\par
\endgroup
\pagebreak

%----------------------------------------------------------------------------------------
%	PEOPLE PAGE
%----------------------------------------------------------------------------------------
\chapterimage{banner3} % Chapter heading image
\begin{center}
	\par
	{\large PRESIDENTE DA REPÚBLICA \\ Jair Messias Bolsonaro}\vspace*{1cm}
	\par
	{\large MINISTRO DA EDUCAÇÃO \\ Abraham Bragança de Vasconcellos Weintraub}\vspace*{1cm}
	\par
	{\large SECRETÁRIO DE EDUCAÇÃO PROFISSIONAL E TECNOLÓGICA \\ Ariosto Antunes Culau}\vspace*{1cm}
	\par
	{\large REITOR DO INSTITUTO FEDERAL DE GOIÁS \\ Jerônimo Rodrigues da Silva}\vspace*{1cm}
	\par
	{\large PRÓ-REITORA DE ENSINO \\ Oneida Cristina Gomes Barcelos Irigon}\vspace*{1cm}
	\par
	{\large COORDENAÇÃO DE ENSINO SUPERIOR \\ Vinicius Sousa Ferreira}\vspace*{1cm}
	\par
	{\large COORDENADOR DO CURSO \\ Vinícius Gomes Ferreira}\vspace*{1cm}
\end{center}

\chapterimage{banner3} % Chapter heading image
\renewcommand\contentsname{Sumário}
\tableofcontents

%----------------------------------------------------------------------------------------
%	CHAPTER
%----------------------------------------------------------------------------------------
\chapterimage{01.jpg} % Chapter heading image
\chapter{Apresentação}\label{apresentacao}
\vspace{6em}
\begin{flushright}
	\textit{\textcolor{white}{Um bonita citação...}}
\end{flushright}
\vspace{12em}

\todo[inline]{NNNNNNNNNNNNN}~\parencite{Resolucao3De2002}


%-----------------------------------------------
\newpage  
%------------------------------------------------
\section{Identificação do Curso}
\begin{itemize}
	\item \textbf{Instituição Ofertante:} Instituto Federal de Educação, Ciência e Tecnologia de Goiás
	\item \textbf{Nome do curso:} Tecnologia em Análise e Desenvolvimento de Sistemas
	\item \textbf{Carga Horária do Curso:} 2160 horas
	\item \textbf{Forma de oferta:} Presencial
	\item \textbf{Duração:} 3 anos
	\item \textbf{Número de Vagas:} 30 vagas anuais
	\item \textbf{Local de Oferta:} Instituto Federal de Goiás - Câmpus Formosa
	\item \textbf{Reitor:} Jerônimo Rodrigues da Silva
	\item \textbf{Pró-Reitora de Ensino:} Oneida Cristina Gomes Barcelos Irigon
	\item \textbf{Coordenação de Ensino Superior:} Vinicius Sousa Ferreira
\end{itemize}

\section{Elaboração do Projeto de Curso}
\begin{itemize}[label=\bfseries]
	\item Danilo Souza de Almeida
	\item Eliana Carla Rodrigues
	\item Vinícius Gomes Ferreira
	\item Waldeyr Mendes Cordeiro da Silva
\end{itemize}

%----------------------------------------------------------------------------------------
%	CHAPTER
%----------------------------------------------------------------------------------------
\chapterimage{02.jpg} % Chapter heading image
\chapter{Introdução}\label{introducao}
\vspace{6em}
\begin{flushright}
	\textit{\textcolor{white}{Um bonita citação...}}
\end{flushright}
\vspace{12em}

A justificativa por trás da implantação e ampliação dos Cursos Superiores de Tecnologia se refere ao fato de que as necessidades econômicas e profissionais da sociedade são dinâmicas e, por isso, é importante que as instituições de ensino superior (IES) brasileiras possam disponibilizar o quanto antes um contingente de profissionais que possa antedê-la. A proposta de implantação e oferta do curso superior de Tecnologia em Análise e Desenvolvimento de Sistemas (TADS) surge, portanto, com a intenção de suprir as necessidades locais, regionais e nacionais de um setor que cresce muito nos últimos anos e possui prognósticos de revolucionar o modo como a economia e as relações humanas são conduzidas na vida em sociedade, o que está alinhado com os objetivos contidos no Plano de Desenvolvimento Institucional do Instituto Federal de Educação, Ciência e Tecnologia de Goiás (IFG).

O arcabouço legal que apoia, regulamenta, ampara e dá as bases de funcionamento para os cursos superiores de tecnologia conta com a Lei nº 9.394 (Lei de Diretrizes e Bases da Educação Nacional -
LDB), em 20 de dezembro de 1996, com o Decreto nº 5.154, de
23 de julho de 2004, que regulamentou os artigos da LDB referentes à educação profissional e tecnológica, pelo Parecer CNE/CES nº 436/01, de 02 de abril de 2001, que trata de Cursos Superiores de Tecnologia – Formação de Tecnólogos, e pela Resolução CNE/CP nº 3, de 18 de dezembro de 2002, que institui as Diretrizes Curriculares Nacionais Gerais para a Educação Profissional de Nível Tecnológico. Seus princípios versam sobre uma proposta de caracterização de
um modelo ensino superior que se atenta às demandas de um mundo do trabalho competitivo, dinâmico e que está em constante mutação, criando, principalmente, meios para que o tempo de formação seja reduzido e os conteúdos ensinados estejam alinhados com a realidade vigente no país e no mundo, de acordo com os requisitos profissionais de cada área de atuação. 

Para que isso se tornasse praticável entre as IES, no tocante ao cumprimento do Decreto nº 5.773/06, o Ministério da Educação trouxe em 2006 o Catálogo Nacional de Cursos Superiores de Tecnologias, que serve como ponto de referência para a comunidade educacional composta por estudantes, educadores, instituições, sistemas e redes de ensino, entidades representativas de classes, empregadores e o público em geral e que  organiza e orienta a oferta de cursos superiores de tecnologia. Este presente Projeto Pedagógico de Curso estabelece uma proposta de Curso Superior de Tecnologia em Análise e Desenvolvimento de Sistemas à luz deste catálogo, como também dos princípios estabelecidos em lei e das necessidades locais e regionais presentes no município de Formosa e regiões circunvizinhas.

\section{Justificativa}

A área de desenvolvimento de software tem sido, desde algum tempo, estratégica para a ampla maioria das empresas e também para o progresso das nações e da sociedade em geral \cite{sommerville2019engenharia}. Se trata de um progresso surpreendente alcançado, sobretudo nos últimos 50 anos  \cite{sommerville2019engenharia}. Vivemos em uma era que, desde serviços de utilidade pública, até a construção e manutenção de infraestrutura nacional depende de sistemas informatizados. Além disso, há desafios com os quais a humanidade deve lidar nos dias de hoje, que exigem a atualização de tecnologias nas quais o software exerce papel fundamental \cite{sommerville2019engenharia}. Esta é uma era em que os profissionais de desenvolvimento de software se tornaram - e hão de continuar a ser - aqueles na categoria dos mais procurados pela indústria em geral \cite{pressman2016engenharia}.

Portanto, a carreira em desenvolvimento de software apresenta-se promissora no cenário global, nacional e regional. Para o estado de Goiás, o cenário mostra-se bastante promissor com a indústria do setor de softwares e serviços dando sinais de estar em franca acensão \cite{Empreendeder2017}. Muitas tecnologias baseadas vêm ganhando espaço, tornando-se necessárias para as demandas comerciais, industriais e de serviço público, exigindo um contingente cada vez maior de profissionais treinados para lidar com elas e, sobretudo, ajudar a projetá-las \cite{Istoe2019}. Segundo dados da Brasscom, há 845 mil empregos no segmento de Tecnologia da Informação (TI) no Brasil e projeta-se necessidade de mais 70 mil profissionais nessa área até 2024 \cite{Convergencia2019}. Embora o país passe por problemas de desemprego, há ainda muita carência de profissionais para o setor, com 5 mil vagas abertas apenas em startups, empresas de base tecnológica alancadas pela transformação digital e que são frutos da nova economia. Esse número tende a crescer ainda mais \cite{Brasscom2019}. Além disso, os negócios mais maduros também experimentam esse crescimento, que segundo o mesmo estudo, pode chegar à receita de R\$ 200 bilhões até 2024 \cite{Brasscom2019}.

Incluído dentro da área voltada para as tecnologias de informação, o mercado brasileiro voltado unicamente para softwares e serviços se destaca, aparecendo entre os 10 primeiros com mais investimento \cite{Abes2018}. O estudo Mercado Brasileiro de Software – Panorama e Tendência 2019 aponta que, no geral, o Brasil possui um mercado interno total de Tecnologia da Informação de US\$ 46.637 milhões, tendo o setor de software uma parcela de US\$ 10.479 milhões desse mercado \cite{Abes2018}. Já a região centro-oeste está entre as três regiões com maior distribuição do mercado brasileiro de software \cite{Abes2018}. Nessa região, o Distrito Federal e o estado de Goiás ocupam o primeiro e terceiro lugar, respectivamente e distribuição do mercado de software, em milhões de US\$, estados que compreendem, influenciam e sofrem influência da região onde está estabelecida a cidade de Formosa \cite{Abes2018}.

É notável que nas três últimas décadas, a dinâmica da economia mundial sofreu profundas transformações nos modelos de geração e acumulação de riqueza. Diferentemente do antigo padrão de acumulação baseado em recursos tangíveis, dispersos ao redor do mundo, no atual padrão, o conhecimento e a informação exercem papeis centrais, sendo as tecnologias de informação e comunicação seu elemento propulsor. Formosa e cidades circunvizinhas, que possuem sua economia baseada na agropecuária, na prestação de serviços e na administração pública, podem ser beneficiadas pela formação de profissionais capazes de lidar com as questões cuja origem parte da transformação digital que atinge muitas áreas, sobretudo estas três.


\subsection{Características socieconômicas do município de Formosa}

Formosa-GO é um município goiano, fundado em 1 de agosto de 1843, com população aproximada estimada em 119 mil pessoas em 2018. Faz fronteira, a oeste, com o Distrito Federal, possui uma área de 5.811,8 km², com densidade geográfica de 17,22 habitantes por Km² e um PIB per capita de R\$ 18.456,69. 

A renda média do formosense é de 1,9 salários mínimos e a proporção de pessoas ocupadas em relação à população total é de 15,4\%. Cerca de 34\% da população de Formosa vive em domicílios cuja a renda individual é de meio salário mínimo ou menos. Quase metade da população (47,26\%) não tem ocupação econômica, enquanto, entre a população ocupada, 13,09\% têm sua carteira de trabalho assinada.
A escolarização da população entre 6 e 14 anos de idade é de 96\% com IDEB de 5,2 para os anos iniciais do ensino fundamental e 4,2 para os anos finais. Cerca de 4,3 mil estudantes estão matriculados no ensino médio distribuídos em cerca 22 escolas ofertantes regulares. No ensino superior, há aproximadamente 2,5 mil estudantes distribuídos em 22 cursos presenciais ofertados. Há na mesorregião do leste goiano, onde situa-se Formosa, 0,69 doutores e 1,81 mestres para cada 100 mil habitantes.

Há 30 estabelecimentos de saúde em Formosa, cuja taxa de mortalidade infantil é de 5,45 óbitos por mil nascidos vivos. O acesso aos serviços públicos de saúde alcança aproximadamente 93\%  da população de Formosa e cerca de 7\% dos residentes não utilizam este tipo de atendimento. O esgotamento sanitário adequado está acessível para 42,9\% da população.

\section{Público Alvo}

O curso Superior de Tecnologia em Análise e Desenvolvimento de Sistemas é voltado para jovens e adultos que desejam ingressar na área de Tecnologia da Informação, com enfoque em desenvolvimento de sistemas e também em ciência de dados. Espera-se que os jovens e adultos habitantes da cidade de Formosa e das cidades circunvizinhas possam se beneficiar da existência do curso e de sua natureza pública e gratuita a fim de que possam pleitear vagas de emprego ou montar empreendimentos de base tecnológica na região onde habitam, nas cidades do Distrito Federal, inclusive Brasília, ou, em última análise, em qualquer região do Brasil e do mundo. 


Sua matriz foi desenhada de forma que possa capacitar os alunos matriculados no curso a lidarem com os desafios de uma sociedade que depende cada vez mais de software e que tem cada vez mais dado valor ao grande volume de dados gerados pelos softwares já foram ou ainda serão desenvolvidos.

\section{Objetivos}\label{objetivos}

\subsection{Objetivo Geral}

\todo[inline]{NNNNNNNNNNNNN}

\subsection{Objetivos Específicos}

\begin{itemize}
\item NNNNNNNNNNNNN 
\end{itemize}


\section{Perfil do Egresso}
O Tecnólogo em Análise e Desenvolvimento de Sistemas:
\begin{enumerate}
	\item Analisa, projeta, desenvolve, testa, implanta e mantém sistemas computacionais de informação. 
	\item Avalia, seleciona, especifica e utiliza metodologias, tecnologias e ferramentas da Engenharia de Software, linguagens de programação e bancos de dados. 
	\item Coordena ou participa de times de desenvolvimento de softwares trabalhando em equipe.
	\item Compreende os fundamentos científico-tecnológicos e a importância e impacto do seu trabalho.
	\item Respeita as diversidades e os direitos humanos com atitude ética e responsabilidade sócio-ambiental no trabalho e fora dele.
	\item Usa a linguagem para a cidadania e profissão.
	\item Conhece planejamento estratégico, iniciativa e liderança.
	\item Atualiza-se mantendo-se criativo e responsável.
\end{enumerate}

%----------------------------------------------------------------------------------------
%	CHAPTER
%----------------------------------------------------------------------------------------
\chapterimage{03.jpg} % Chapter heading image
\chapter{Organização do Curso}\label{organizacao}
\vspace{6em}
\begin{flushright}
	\textit{\textcolor{white}{Um bonita citação...}}
\end{flushright}
\vspace{12em}

\section{Requisitos para Acesso ao Curso}

\todo[inline]{Prever entrada via vestibular, ENEM e SISU; portador de diploma, transferência interna e externa nos termos dos regulamentos do IFG;}

\section{Forma de Oferta}\label{carga}

\todo[inline]{EaD; Matutino ou Noturno;}

\section{Metodologia de Ensino-Aprendizagem}\label{metodologia}

\todo[inline]{Dissertar sobre a questao dos projetos e semestres tematicos;}

\section{Matriz Curricular}\label{matriz}

\todo[inline]{NNNNNNNNNNNNN}
%\caption{Matriz Curricular do Curso Superior em Tecnologia em Análise e Desenvolvimento de Sistemas.}

% Please add the following required packages to your document preamble:
% \usepackage{multirow}
% \usepackage{graphicx}
\begin{table}[]
	\centering
	\caption{Matriz Curricular do Curso Superior de Tecnologia em Análise e Desenvolvimento de Sistemas.}
	\label{tab:matriz}
	\resizebox{\textwidth}{!}{%
		\begin{tabular}{|l|l|c|c|c|c|c|c|c|}
			\hline
			\textbf{Tema}                                                    
			& \textbf{Disciplina}                                           & \textbf{1º Sem} & \textbf{2º Sem} & \textbf{3º Sem} & \textbf{4º Sem} & \textbf{5º Sem} & \textbf{6º Sem} & \textbf{CH}          \\ \hline
			\multirow{6}{*}{Fundamentos da ADS} 
			& \nameref{1_algoritmos}                                        & 54              &                 &                 &                 &                 &                 & \multirow{6}{*}{270} \\ \cline{2-8}
			& \nameref{1_fundcomp}                                          & 54              &                 &                 &                 &                 &                 &                      \\ \cline{2-8}
			& \nameref{1_engsof}                                            & 27              &                 &                 &                 &                 &                 &                      \\ \cline{2-8}
			& \nameref{1_matematica}                                        & 54              &                 &                 &                 &                 &                 &                      \\ \cline{2-8}
			& \nameref{1_leitprodtextos}                                    & 27              &                 &                 &                 &                 &                 &                      \\ \cline{2-8}
			& \nameref{1_inglacad}                                          & 54              &                 &                 &                 &                 &                 &                      \\ \hline
			\multirow{5}{*}{Fundamentos da ADS} 
			& \nameref{2_sistop}                                            &                 & 54              &                 &                 &                 &                 & \multirow{5}{*}{270} \\ \cline{2-8}
			& \nameref{2_bancodados}                                        &                 & 54              &                 &                 &                 &                 &                      \\ \cline{2-8}
			& \nameref{2_arqsoft}                                           &                 & 54              &                 &                 &                 &                 &                      \\ \cline{2-8}
			& \nameref{2_estruturadedados}                                  &                 & 54              &                 &                 &                 &                 &                      \\ \cline{2-8}
			& \nameref{2_algebra}                                           &                 & 54              &                 &                 &                 &                 &                      \\ \hline
			\multirow{7}{*}{Meio Ambiente e Informática}                     
			& \nameref{3_poo}                                               &                 &                 & 27              &                 &                 &                 & \multirow{7}{*}{270} \\ \cline{2-8}
			& \nameref{3_nosql}                                             &                 &                 & 27              &                 &                 &                 &                      \\ \cline{2-8}			
			& \nameref{3_testsoft}                                          &                 &                 & 27              &                 &                 &                 &                      \\ \cline{2-8}
			& \nameref{3_engreq}                                            &                 &                 & 54              &                 &                 &                 &                      \\ \cline{2-8}
			& \nameref{3_redescomp}                                         &                 &                 & 54              &                 &                 &                 &                      \\ \cline{2-8}
			& \nameref{3_educamb}                                           &                 &                 & 27              &                 &                 &                 &                      \\ \cline{2-8}
			&\nameref{3_projamb}                                            &                 &                 & 54              &                 &                 &                 &                      \\ \hline
			\multirow{5}{*}{Sociedade e Informática}                         
			& \nameref{4_ppw1}                                              &                 &                 &                 & 54              &                 &                 & \multirow{6}{*}{270} \\ \cline{2-8}
			& \nameref{4_ihc}                                               &                 &                 &                 & 27              &                 &                 &                      \\ \cline{2-8}
			& \nameref{4_asi}                                               &                 &                 &                 & 54              &                 &                 &                      \\ \cline{2-8}
			& \nameref{4_probest}                                           &                 &                 &                 & 54              &                 &                 &                      \\ \cline{2-8}
			& \nameref{4_etnicoraciais}                                     &                 &                 &                 & 27              &                 &                 &                      \\ \cline{2-8}
			& \nameref{4_projsoc}                                           &                 &                 &                 & 54              &                 &                 &                      \\ \hline
			\multirow{6}{*}{Inclusão e Informática}                          
     		& \nameref{5_ppw2}                                              &                 &                 &                 &                 & 54              &                 & \multirow{6}{*}{270} \\ \cline{2-8}
			& \nameref{5_metodologia}                                       &                 &                 &                 &                 & 54              &                 &                      \\ \cline{2-8}
			& \nameref{5_libras}                                            &                 &                 &                 &                 & 27              &                 &                      \\ \cline{2-8}
			& \nameref{5_etica}                                             &                 &                 &                 &                 & 27              &                 &                      \\ \cline{2-8}
			& \nameref{5_opt}                                               &                 &                 &                 &                 & 54              &                 &                      \\ \cline{2-8}
			& \nameref{5_lab}                                               &                 &                 &                 &                 & 54              &                 &                      \\ \hline
			\multirow{6}{*}{Mercado e Informática}                           
			& \nameref{6_ia}                                             &                 &                 &                 &                 &                 & 54              & \multirow{6}{*}{270} \\ \cline{2-8}
			& \nameref{6_seginfo}                                        &                 &                 &                 &                 &                 & 27              &                      \\ \cline{2-8}
			& \nameref{6_empdig}                                         &                 &                 &                 &                 &                 & 27              &                      \\ \cline{2-8}
			& \nameref{6_datascience}                                    &                 &                 &                 &                 &                 & 54              &                      \\ \cline{2-8}
			& \nameref{6_visstory}                                       &                 &                 &                 &                 &                 & 54              &                      \\ \cline{2-8}
			& \nameref{6_govproj}                                        &                 &                 &                 &                 &                 & 54              &                      \\ \hline
			\multicolumn{8}{|l|}{Carga Horária Total de Disciplinas (Hora Relógio)}                                                                                                                                                                                     
			& 1620                 \\ \hline
			\multicolumn{8}{|l|}{Carga Horária Total de Disciplinas (Hora Aula)}                                                                                                                                                                                        
			& 2160                 \\ \hline
			\multicolumn{8}{|l|}{Carga Horária Atividades Complementares (Hora Relógio)}                                                                                                                                                                                     
			& 400                 \\ \hline
			\multicolumn{8}{|l|}{Carga Horária Total do Curso (Hora Relógio)}                                                                                                                                                                                        
			& 2020                 \\ \hline
			\multicolumn{8}{|l|}{Carga Horária Total do Curso (Hora Aula)}                                                                                                                                                                                        
			& 2666                 \\ \hline			
		\end{tabular}%
	}
\end{table}

\subsubsection{Atividades Complementares}

As Atividades Complementares são atividades relacionadas ao curso organizadas em um componente curricular.
Portanto, são exigidas como um requisito parcial para a formação dos alunos na pós-graduação.
Elas devem ser desenvolvidas pelo aluno regular, paralelamente às demais disciplinas acadêmicas nos termos da Resolução 16, de 26 de Dezembro de 2011~\cite{Resolucao16De2011}.
Para isso, a carga horária das atividades será concedida de acordo com os itens abaixo:

\begin{enumerate}
	\item Visitas Técnicas.
	\item Atividades Práticas de Campo.
	\item Participação em eventos técnicos, científicos, acadêmicos, culturais, artísticos e esportivos.
	\item Participação em comissão organizadora de eventos institucionais e outros.
	\item Apresentação de trabalhos em feiras, congressos, mostras, seminários e outros.
	\item Intérprete de línguas em eventos institucionais e outros.
	\item Monitorias por período mínimo de um semestre letivo.
	\item Participação em projetos e programas de iniciação científica e tecnológica como aluno titular do projeto, bolsista ou voluntário.
	\item Participação em programa de iniciação a docência como aluno bolsista ou voluntário.
	\item Participação em projetos de ensino, pesquisa e extensão com duração mínima de um semestre letivo.
	\item Cursos e minicursos.
	\item Estágio curricular não obrigatório igual ou superior a cem horas.
	\item Participação como representante de turma por um período mínimo de um semestre letivo.
	\item Participação como representante discente nas instâncias da Instituição por um período mínimo de um semestre letivo.
	\item Participação em órgãos e entidades estudantis, de classe, sindicais ou comunitárias.
	\item Realização de trabalho comunitário.
	\item Participação como ouvinte em defesas de trabalhos acadêmicos.
\end{enumerate}


\subsubsection{Avaliação}

\todo[inline]{NNNNNNNNNNNNN}

\section{Certificação}

O Certificado será emitido pelo Instituto Federal de Educação, Ciência e Tecnologia de Goiás, nos termos da Resolução CNE/CES nº 1, de 8 de junho de 2007.	
Para obter o Certificado de graduado em ``Tecnologia de Análise e Desenvolvimento de Sistemas'', o discente deverá satisfazer as seguintes exigências:
\begin{enumerate}
	\item Ser aprovado em todas as disciplinas do curso com nota mínima igual a 6,0 (seis) e freqüência igual ou superior a 75\% da carga horária;
	\item Ser aprovado em defesa pública do Trabalho de Conclusão de Curso (TCC) perante uma banca composta por, no mínimo, três professores (o orientador e dois ou mais convidados);
	\item Possuir pelo menos um certificado que comprove a apresentação (pôster ou oral) de resultados relacionados ao TCC em evento científico (congressos, seminários, simpósios);
	\item Cumprir a carga horária de Atividades Complementares prevista neste Projeto Pedagógico de Curso;
	\item Quitação de todas as obrigações junto ao Câmpus Formosa do Instituto Federal de Educação, Ciência e Tecnologia de Goiás;
\end{enumerate}








%----------------------------------------------------------------------------------------
%	CHAPTER
%----------------------------------------------------------------------------------------
%------------------------------------------------
\chapterimage{04.jpg} % Chapter heading image
\chapter{Ementas}\label{ementas}
\vspace{6em}
\begin{flushright}
	\textit{\textcolor{white}{Um bonita citação...}}
\end{flushright}
\vspace{12em}

%%%%%%%%%%%%%%%%%%%%%%%%%%%%%%%%%%%%%%%%%%%%%%%%%%%%%%%%%%%%%%%%%%%%%%%%%%%%%%
%%%%%%%%%%%%%%%%%%%%%%%%%%%%%%%%%%%%%%%%%%%%%%%%%%%%%%%%%%%%%%%%%%%%%%%%%%%%%%
%%%% 1 SEMESTRE
%%%%%%%%%%%%%%%%%%%%%%%%%%%%%%%%%%%%%%%%%%%%%%%%%%%%%%%%%%%%%%%%%%%%%%%%%%%%%%
%%%%%%%%%%%%%%%%%%%%%%%%%%%%%%%%%%%%%%%%%%%%%%%%%%%%%%%%%%%%%%%%%%%%%%%%%%%%%%


%%%%%%%%%%%%%%%%%%%%%%%%%%%%%%%%%%%%%%%%%%%%%%%%%%%%%%%%%%%%%%%%%%%%%%%%%%%%%%
%%%%  Algoritmos
%%%%%%%%%%%%%%%%%%%%%%%%%%%%%%%%%%%%%%%%%%%%%%%%%%%%%%%%%%%%%%%%%%%%%%%%%%%%%%
\newpage
\section{Algoritmos}\label{1_algoritmos}
\begin{itemize}
	\item \textbf{Carga horária (hora/aula):} 54
	\item \textbf{Objetivo:} Proporcionar habilidades fundamentais em programação de computadores.
	\item \textbf{Ementa:} 
	Conceitos de algoritmos;
	Conceitos de linguagens de programação;
	Constantes e variáveis;
	Tipos de dados;
	Operadores e expressões aritméticas, lógicas e literais; 
	Comandos básicos;
	Estruturas condicionais e de repetição;
	Vetores e matrizes;
	Estruturas de dados básicas;
	Modularização;
	Recursividade;
	Algoritmos e meio ambiente;
	\item \textbf{Bibliografia básica}
	\begin{enumerate}
		\item \cite{cormen2002algoritmos}
		\item \cite{silva2007estrutura}
		\item \cite{szwarcfiter1994estruturas}
	\end{enumerate}
	\item \textbf{Bibliografia complementar}
	\begin{enumerate}
		\item \cite{ascencio2010estruturas}
		\item \cite{lafore2004estruturas}
	\end{enumerate}	
\end{itemize}
\nameref{tab:matriz}

%%%%%%%%%%%%%%%%%%%%%%%%%%%%%%%%%%%%%%%%%%%%%%%%%%%%%%%%%%%%%%%%%%%%%%%%%%%%%%
%%%%  Fundamentos da Computação
%%%%%%%%%%%%%%%%%%%%%%%%%%%%%%%%%%%%%%%%%%%%%%%%%%%%%%%%%%%%%%%%%%%%%%%%%%%%%%
\newpage
\section{Fundamentos da Computação}\label{1_fundcomp}
\begin{itemize}
	\item \textbf{Carga horária (hora/aula):} 54
	\item \textbf{Objetivo:}Proporcionar conhecimentos fundamentais em \textit{hardware} e \textit{software}.
	\item \textbf{Ementa:} 
	Introdução aos Sistemas Computacionais: importância da informática na atualidade, dado e informação, conceitos de computador e computação, tipos de dados primitivos, medidas de armazenamento; 
	Histórico dos Computadores: as gerações de computadores; 
	Tipos de Computadores; Arquiteturas RISC e CISC; 
	Sistemas de Numeração: binário, octal, decimal e hexadecimal; 
	Software: conceitos, softwares básicos - BIOS, Drivers e Sistemas Operacionais, softwares aplicativos e utilitários, compiladores e linguagens de programação;
	Hardware: periféricos de E/S, componentes internos: placa-mãe e barramentos, CPU (UC, ULA, registradores, cache), hierarquia de memória, placas de rede, som e vídeo, HD, drives ópticos, fontes e gabinetes;  
	Redes de Computadores e Internet: topologias e protocolos, serviços da Internet; Bancos de Dados: conceitos básicos; Segurança de Sistemas: ameaças, segurança física, segurança lógica e criptografia;
	Resíduos eletrônicos e responsabilidade social e ambiental.
	\item \textbf{Bibliografia básica}
	\begin{enumerate}
		\item 
	\end{enumerate}
	\item \textbf{Bibliografia complementar}
	\begin{enumerate}
		\item 
	\end{enumerate}	
\end{itemize}
\nameref{tab:matriz}


%%%%%%%%%%%%%%%%%%%%%%%%%%%%%%%%%%%%%%%%%%%%%%%%%%%%%%%%%%%%%%%%%%%%%%%%%%%%%%
%%%%  Fundamentos em Engenharia de Software
%%%%%%%%%%%%%%%%%%%%%%%%%%%%%%%%%%%%%%%%%%%%%%%%%%%%%%%%%%%%%%%%%%%%%%%%%%%%%%
\newpage	
\section{Fundamentos em Engenharia de Software}\label{1_engsof}
\begin{itemize}
	\item \textbf{Carga horária (hora/aula):} 27
	\item \textbf{Objetivo:}Proporcionar conhecimentos fundamentais em Engenharia de Software de forma a preparar o estudante a desenvolver \textit{software} segundo boas práticas consolidadas.
	\item \textbf{Ementa:} 
	Software e sua natureza;
	Conceitos de Engenharia de Software;
	Aspectos humanos e éticos em Engenharia de Software;
	O processo genérico de Engenharia de Software;
	Modelos de processo de Engenharia de Software;
	Filosofia ágil de desenvolvimento de software;
	Evolução do software;
	Ferramentas de desenvolvimento de software (CASE);
	Gerenciamento da Configuração;
	\item \textbf{Bibliografia básica}
	\begin{enumerate}
		\item \cite{sommerville2011engenharia}
		\item \cite{pressman2016engenharia}
		\item \cite{de2003engenharia}
	\end{enumerate}
	\item \textbf{Bibliografia complementar}
	\begin{enumerate}
		\item ~\cite{wazlawick2011analise}
		\item \cite{prikladnicki2014metodos}(\textcolor{red}{Não tem na biblioteca do câmpus Formosa.})
		\item \cite{hirama2012engenharia}(\textcolor{red}{Não tem na biblioteca do câmpus Formosa.})
		\item \cite{engholm2010engenharia}(\textcolor{red}{Não tem na biblioteca do câmpus Formosa.})
	\end{enumerate} 
\end{itemize}
\nameref{tab:matriz}


%%%%%%%%%%%%%%%%%%%%%%%%%%%%%%%%%%%%%%%%%%%%%%%%%%%%%%%%%%%%%%%%%%%%%%%%%%%%%%
%%%%  Matemática
%%%%%%%%%%%%%%%%%%%%%%%%%%%%%%%%%%%%%%%%%%%%%%%%%%%%%%%%%%%%%%%%%%%%%%%%%%%%%%
\newpage
\section{Matemática}\label{1_matematica}
\begin{itemize}
	\item \textbf{Carga horária (hora/aula):} 54
	\item \textbf{Objetivo:}Proporcionar uma revisão de conceitos fundamentais em Matemática preparando o estudante para estudos avançados.
	\item \textbf{Ementa:} 
	Conceitos fundamentais de funções;
	Funções: composta, inversa, afim, quadrática, modular, exponencial, logarítmica e trigonométrica;
	Progressões;
	Análise Combinatória.
	\item \textbf{Bibliografia básica}
	\begin{enumerate}
		\item 
	\end{enumerate}
	\item \textbf{Bibliografia complementar}
	\begin{enumerate}
		\item 
	\end{enumerate}	
\end{itemize}
\nameref{tab:matriz}


%%%%%%%%%%%%%%%%%%%%%%%%%%%%%%%%%%%%%%%%%%%%%%%%%%%%%%%%%%%%%%%%%%%%%%%%%%%%%%
%%%%  Leitura e Produção de Textos
%%%%%%%%%%%%%%%%%%%%%%%%%%%%%%%%%%%%%%%%%%%%%%%%%%%%%%%%%%%%%%%%%%%%%%%%%%%%%%
\newpage
\section{Leitura e Produção de Textos}\label{1_leitprodtextos}
\begin{itemize}
	\item \textbf{Carga horária (hora/aula):} 27
	\item \textbf{Objetivo:} Preparar o estudante para exercer a comunicação acadêmica e corporativa em língua portuguesa.
	\item \textbf{Objetivo:} Desenvolver habilidades de leitura, escuta e escrita da língua portuguesa para exercer a comunicação oral e escrita e compreensão de textos acadêmicos.
	\item \textbf{Ementa:} 
	Práticas de leitura, escuta e escrita em língua portuguesa a partir dos conhecimentos prévios em língua portuguesa.
	\item \textbf{Bibliografia básica}
	\begin{enumerate}
		\item 
	\end{enumerate}
	\item \textbf{Bibliografia complementar}
	\begin{enumerate}
		\item 
	\end{enumerate}	
\end{itemize}
\nameref{tab:matriz}

%%%%%%%%%%%%%%%%%%%%%%%%%%%%%%%%%%%%%%%%%%%%%%%%%%%%%%%%%%%%%%%%%%%%%%%%%%%%%%
%%%%  Leitura e Produção de Textos
%%%%%%%%%%%%%%%%%%%%%%%%%%%%%%%%%%%%%%%%%%%%%%%%%%%%%%%%%%%%%%%%%%%%%%%%%%%%%%
\newpage
\section{Leitura e Produção de Textos em Inglês}\label{1_inglacad}
\begin{itemize}
	\item \textbf{Carga horária (hora/aula):} 54
	\item \textbf{Objetivo:}Preparar o estudante para exercer a comunicação acadêmica e corporativa em língua inglesa.
	\item \textbf{Objetivo:} Desenvolver habilidades de leitura, escuta e escrita da língua inglesa para exercer a comunicação oral e escrita e compreensão de textos acadêmicos.
	\item \textbf{Ementa:} 
	Práticas de leitura, escuta e escrita em língua inglesa a partir dos conhecimentos prévios em língua inglesa, com a utilização do suporte da língua portuguesa.
	\item \textbf{Bibliografia básica}
	\begin{enumerate}
		\item 
	\end{enumerate}
	\item \textbf{Bibliografia complementar}
	\begin{enumerate}
		\item 
	\end{enumerate}	
\end{itemize}
\nameref{tab:matriz}

%%%%%%%%%%%%%%%%%%%%%%%%%%%%%%%%%%%%%%%%%%%%%%%%%%%%%%%%%%%%%%%%%%%%%%%%%%%%%%
%%%%%%%%%%%%%%%%%%%%%%%%%%%%%%%%%%%%%%%%%%%%%%%%%%%%%%%%%%%%%%%%%%%%%%%%%%%%%%
%%%%  2 SEMESTRE
%%%%%%%%%%%%%%%%%%%%%%%%%%%%%%%%%%%%%%%%%%%%%%%%%%%%%%%%%%%%%%%%%%%%%%%%%%%%%%
%%%%%%%%%%%%%%%%%%%%%%%%%%%%%%%%%%%%%%%%%%%%%%%%%%%%%%%%%%%%%%%%%%%%%%%%%%%%%%


%%%%%%%%%%%%%%%%%%%%%%%%%%%%%%%%%%%%%%%%%%%%%%%%%%%%%%%%%%%%%%%%%%%%%%%%%%%%%%
%%%%  Sistemas Operacionais
%%%%%%%%%%%%%%%%%%%%%%%%%%%%%%%%%%%%%%%%%%%%%%%%%%%%%%%%%%%%%%%%%%%%%%%%%%%%%%
\newpage
\section{Fundamentos de Sistemas Operacionais}\label{2_sistop}
\begin{itemize}
	\item \textbf{Carga horária (hora/aula):} 54
	\item \textbf{Objetivo:} Proporcionar ao estudante fundamentos teóricos e habilidades práticas em Sistemas Operacionais.
	\item \textbf{Ementa:} 
	Conceito Fundamental de Sistema Operacional; 
	Tipos de Sistemas Operacionais; 
	História dos Sistemas Operacionais; 
	Arquiteturas Notáveis de Sistema Operacional; 
	Processo; 
	Comunicação entre processos; 
	Gerência do Processador; 
	Gerência de Memória; 
	Gerência de Dispositivos; 
	Sistemas de Arquivos; 
	Estudos de casos de sistemas operacionais atuais.
	\item \textbf{Bibliografia básica}
	\begin{enumerate}
		\item 
	\end{enumerate}
	\item \textbf{Bibliografia complementar}
	\begin{enumerate}
		\item 
	\end{enumerate}	
\end{itemize}
\nameref{tab:matriz}


%%%%%%%%%%%%%%%%%%%%%%%%%%%%%%%%%%%%%%%%%%%%%%%%%%%%%%%%%%%%%%%%%%%%%%%%%%%%%%
%%%%  Fundamentos de Bancos de Dados
%%%%%%%%%%%%%%%%%%%%%%%%%%%%%%%%%%%%%%%%%%%%%%%%%%%%%%%%%%%%%%%%%%%%%%%%%%%%%%
\newpage
\section{Fundamentos de Bancos de Dados}\label{2_bancodados}

\begin{itemize}
	\item \textbf{Carga horária (hora/aula):} 54
	\item \textbf{Objetivo:} Proporcionar ao estudante fundamentos teóricos e habilidades práticas em sistemas de bancos de dados com ênfase em bancos de dados relacionais
	\item \textbf{Ementa:} 
	Conceitos básicos;
	Modelos Relacional e Objeto Relacional;
	Modelagem de dados; 
	Agregação, Generalização e Cardinalidade; 
	Linguagem SQL; 
	Dialeto SQL, Linguagem de Definição de Dados (DDL) e Linguagem de Manipulação de Dados (DML);
	SGBD: conceitos, modelagem e gerência;
	Dialetos SQL associados a Sistemas Gerenciadores de Banco de Dados (SGBD);
	Procedimentos Armazenados; 
	Visões; 
	Funções; 
	Aperfeiçoamento e Otimização de Consultas;
	Propriedades ACID;
	Tendências em bancos de dados;
	\item \textbf{Bibliografia básica}
	\begin{enumerate}
		\item~\cite{silberschatz2016}
		\item~\cite{date2004}
		\item~\cite{heuser2009}
	\end{enumerate}
	\item \textbf{Bibliografia complementar}
	\begin{enumerate}
		\item~\cite{sadalage2019}
		\item~\cite{casanova2005}
		\item~\cite{milani2008}
	\end{enumerate}	
\end{itemize}
\nameref{tab:matriz}


%%%%%%%%%%%%%%%%%%%%%%%%%%%%%%%%%%%%%%%%%%%%%%%%%%%%%%%%%%%%%%%%%%%%%%%%%%%%%%
%%%%  Arquitetura e Desenho de Software
%%%%%%%%%%%%%%%%%%%%%%%%%%%%%%%%%%%%%%%%%%%%%%%%%%%%%%%%%%%%%%%%%%%%%%%%%%%%%%
\newpage
\section{Arquitetura e Desenho de Software}\label{2_arqsoft}
\begin{itemize}
	\item \textbf{Carga horária (hora/aula):} 54
	\item \textbf{Objetivo:} Proporcionar ao estudante fundamentos teóricos e práticos em arquitetura e desenho de software.
	\item \textbf{Ementa:} 
	Conceitos de Arquitetura e Modelagem de Software;
	Atributos de Qualidade;
	Padrões macro-arquiteturais (estruturas, estilos e visões);
	Padrões micro-arquiteturais (\textit{Design Patterns});
	Documentação de arquitetura de software (\textit{Unified Modeling Language} e outros);
	Arquitetura de software para projetos ágeis;
	Arquitetura no ciclo de vida de software (requisitos, modelagem, implementação, teste, evolução, reconstrução de legados e governança);
	Considerações práticas;
	Normas e padrões pertinentes;
	\item \textbf{Bibliografia básica}
	\begin{enumerate}
		\item ~\cite{sommerville2011engenharia}
		\item ~\cite{pressman2016engenharia}
		\item ~\cite{booch2012uml}
	\end{enumerate}
	\item \textbf{Bibliografia complementar}
	\begin{enumerate}
		\item ~\cite{wazlawick2011analise}
		\item ~\cite{larman2007utilizando}
		\item ~\cite{prikladnicki2014metodos} (\textcolor{red}{Não tem na biblioteca do câmpus Formosa.})
	\end{enumerate}
\end{itemize}
\nameref{tab:matriz}


%%%%%%%%%%%%%%%%%%%%%%%%%%%%%%%%%%%%%%%%%%%%%%%%%%%%%%%%%%%%%%%%%%%%%%%%%%%%%%
%%%%  Estruturas de Dados
%%%%%%%%%%%%%%%%%%%%%%%%%%%%%%%%%%%%%%%%%%%%%%%%%%%%%%%%%%%%%%%%%%%%%%%%%%%%%%
\newpage
\section{Estruturas de Dados}\label{2_estruturadedados}
\begin{itemize}
	\item \textbf{Carga horária (hora/aula):} 54
	\item \textbf{Objetivo:} Proporcionar ao estudante fundamentos teóricos em estruturas de dados homogêneas e heterogêneas, e habilidades práticas através do emprego de linguagem de programação.
	\item \textbf{Ementa:} 
	Análise de algoritmos; 
	Elementos de notação assintótica;
	Algoritmos de ordenação e busca;
	Estruturas de dados homogêneas e heterogêneas;
	Listas;
	Pilhas;
	Filas;
	Tabelas hashing;
	Árvores;
	Grafos;
	Busca em grafos;	
	\item \textbf{Bibliografia básica}
	\begin{enumerate}
		\item \cite{ascencio2010estruturas}
		\item \cite{cormen2002algoritmos}
		\item \cite{silva2007estrutura}
	\end{enumerate}
	\item \textbf{Bibliografia complementar}
	\begin{enumerate}
		\item \cite{szwarcfiter1994estruturas}	
		\item \cite{lafore2004estruturas}	
	\end{enumerate}	
\end{itemize}
\nameref{tab:matriz}

%%%%%%%%%%%%%%%%%%%%%%%%%%%%%%%%%%%%%%%%%%%%%%%%%%%%%%%%%%%%%%%%%%%%%%%%%%%%%%
%%%% Introdução à Álgebra Linear
%%%%%%%%%%%%%%%%%%%%%%%%%%%%%%%%%%%%%%%%%%%%%%%%%%%%%%%%%%%%%%%%%%%%%%%%%%%%%%
\newpage
\section{Introdução à Álgebra Linear}\label{2_algebra}
\begin{itemize}
	\item \textbf{Carga horária (hora/aula):} 54
	\item \textbf{Objetivo:} Apresentar ao estudante os fundamentos da Álgebra Linear e suas aplicações computacionais.
	\item \textbf{Ementa:} 
	Introdução à Álgebra Linear;
	Sistemas de Equações Lineares e Matrizes;
	Determinantes;
	Espaços Vetoriais Euclidianos;
	Aplicações da Álgebra Linear;
	\item \textbf{Bibliografia básica}
	\begin{enumerate}
		\item 
	\end{enumerate}
	\item \textbf{Bibliografia complementar}
	\begin{enumerate}
		\item 	
	\end{enumerate}	
\end{itemize}
\nameref{tab:matriz}



%%%%%%%%%%%%%%%%%%%%%%%%%%%%%%%%%%%%%%%%%%%%%%%%%%%%%%%%%%%%%%%%%%%%%%%%%%%%%%
%%%%%%%%%%%%%%%%%%%%%%%%%%%%%%%%%%%%%%%%%%%%%%%%%%%%%%%%%%%%%%%%%%%%%%%%%%%%%%
%%%%  3 SEMESTRE
%%%%%%%%%%%%%%%%%%%%%%%%%%%%%%%%%%%%%%%%%%%%%%%%%%%%%%%%%%%%%%%%%%%%%%%%%%%%%%
%%%%%%%%%%%%%%%%%%%%%%%%%%%%%%%%%%%%%%%%%%%%%%%%%%%%%%%%%%%%%%%%%%%%%%%%%%%%%%


%%%%%%%%%%%%%%%%%%%%%%%%%%%%%%%%%%%%%%%%%%%%%%%%%%%%%%%%%%%%%%%%%%%%%%%%%%%%%%
%%%%  Programação Orientada a Objetos
%%%%%%%%%%%%%%%%%%%%%%%%%%%%%%%%%%%%%%%%%%%%%%%%%%%%%%%%%%%%%%%%%%%%%%%%%%%%%%
\newpage
\section{Programação Orientada a Objetos}\label{3_poo}
\begin{itemize}
	\item \textbf{Carga horária (hora/aula):} 27
	\item \textbf{Docente Responsável:}~\nameref{ViniciusGomes}
	\item \textbf{Ementa:} 
	Introdução à Programação Orientada a Objetos; 
	Classes e Métodos; 
	Encapsulamento e Sobrecarga; 
	Sobreposição de Métodos; 
	Construtores e Destrutores;
	Herança; 
	Polimorfismo e Ligação Dinâmica; 
	Serialização de Objetos; 
	Programação com threads; 
	Tratamento de exceções;
	Padrões de Projetos Orientados a Objetos.
    
	\item \textbf{Bibliografia básica}
	\begin{enumerate}
		\item 
	\end{enumerate}
	\item \textbf{Bibliografia complementar}
	\begin{enumerate}
		\item 	
	\end{enumerate}	
\end{itemize}

\nameref{tab:matriz}


%%%%%%%%%%%%%%%%%%%%%%%%%%%%%%%%%%%%%%%%%%%%%%%%%%%%%%%%%%%%%%%%%%%%%%%%%%%%%%
%%%%  Big Data e Bancos de Dados NoSQL
%%%%%%%%%%%%%%%%%%%%%%%%%%%%%%%%%%%%%%%%%%%%%%%%%%%%%%%%%%%%%%%%%%%%%%%%%%%%%%
\newpage
\section{Big Data e Bancos de Dados NoSQL}\label{3_nosql}
\begin{itemize}
	\item \textbf{Carga horária (hora/aula):} 27
	\item \textbf{Objetivo:} Proporcionar ao estudante fundamentos teóricos e habilidades práticas em Big data e bancos de dados NoSQL.
	\item \textbf{Ementa:} 
	Big Data: conceito, fontes, tipos de dados;
	Modelagem, tratamento de dados;	
	Bancos de dados NoSQL: bancos de dados em grafos; bancos de dados orientados a documentos; bancos de dados orientados a colunas; chave-valor;
	\item \textbf{Bibliografia Básica}
	\begin{enumerate}
		\item 
		\item 
		\item 
	\end{enumerate}
	\item \textbf{Bibliografia complementar}
	\begin{enumerate}
		\item  
		\item
	\end{enumerate} 	
\end{itemize}
\nameref{tab:matriz}

%%%%%%%%%%%%%%%%%%%%%%%%%%%%%%%%%%%%%%%%%%%%%%%%%%%%%%%%%%%%%%%%%%%%%%%%%%%%%%
%%%%  Técnicas de Engenharia de Requisitos
%%%%%%%%%%%%%%%%%%%%%%%%%%%%%%%%%%%%%%%%%%%%%%%%%%%%%%%%%%%%%%%%%%%%%%%%%%%%%%
\newpage
\section{Fundamentos de Engenharia de Requisitos}\label{3_engreq}
\begin{itemize}
	\item \textbf{Carga horária (hora/aula):} 27
	\item \textbf{Docente Responsável:} Apresentar ao estudante os fundamentos teóricos aliados à prática na elucidação de requisitos de software.
	\item \textbf{Ementa:} 
	Conceito de requisito e necessidade;
	Taxonomia e classificação de requisitos;
	Técnicas de elicitação de requisitos (5W2H, etnografia, entrevista, \textit{workshop}, outros);
	Processo de Engenharia de Requisitos;
	Requisitos em ambientes ágeis (\textit{user stories} e outros);
	Tendências em Engenharia de Requisitos (\textit{Design Thinking}, \textit{Job stories}, \textit{Behavior driven development} e outros)
	\item \textbf{Bibliografia básica}
	\begin{enumerate}
		\item ~\cite{sommerville2011engenharia}
		\item ~\cite{pressman2016engenharia}
		\item ~\cite{prikladnicki2014metodos} (\textcolor{red}{Não tem na biblioteca do câmpus Formosa.})
	\end{enumerate}
	\item \textbf{Bibliografia complementar}
	\begin{enumerate}
		\item ~\cite{wazlawick2011analise}
		\item 
	\end{enumerate}
\end{itemize}
\nameref{tab:matriz}


%%%%%%%%%%%%%%%%%%%%%%%%%%%%%%%%%%%%%%%%%%%%%%%%%%%%%%%%%%%%%%%%%%%%%%%%%%%%%%
%%%%  Técnicas de Engenharia de Requisitos
%%%%%%%%%%%%%%%%%%%%%%%%%%%%%%%%%%%%%%%%%%%%%%%%%%%%%%%%%%%%%%%%%%%%%%%%%%%%%%
\newpage
\section{Redes de Computadores}\label{3_redescomp}
\begin{itemize}
	\item \textbf{Carga horária (hora/aula):} 54
	\item \textbf{Objetivo:} Apresentar ao estudante os fundamentos teóricos e práticos em redes de computadores.
	\item \textbf{Ementa:} 
	Introdução à redes; 
	Tipos de redes e tipos de servidores na topologia cliente-servidor; Classificação e Componentes de uma rede;
	Transmissão de dados;
	Protocolos; 
	Modelo OSI e modelo TCP/IP; 
	Endereçamento IPv4 e IPv6; 
	Cabeamento: Cabo Coaxial, Par Trançado e Fibra Óptica;
	Redes sem Fio;
	Arquiteturas de Redes Locais: Ethernet, Token Ring, FDDI. Equipamentos de Redes: Repetidores e Hubs, Pontes e Switches e Roteadores;
	Tendências em redes de computadores;
	Sustentabilidade e Meio Ambiente: protocolos eficientes para comunicação em redes sem fio.
	\item \textbf{Bibliografia básica}
	\begin{enumerate}
		\item 
	\end{enumerate}
	\item \textbf{Bibliografia complementar}
	\begin{enumerate}
		\item
		\item 
	\end{enumerate}
\end{itemize}
\nameref{tab:matriz}

%%%%%%%%%%%%%%%%%%%%%%%%%%%%%%%%%%%%%%%%%%%%%%%%%%%%%%%%%%%%%%%%%%%%%%%%%%%%%%
%%%%  Verificação, Validação e Teste de Software
%%%%%%%%%%%%%%%%%%%%%%%%%%%%%%%%%%%%%%%%%%%%%%%%%%%%%%%%%%%%%%%%%%%%%%%%%%%%%%
\newpage
\section{Verificação, Validação e Teste de Software}\label{3_testsoft}
\begin{itemize}
	\item \textbf{Carga horária (hora/aula):} 27
	\item \textbf{Docente Responsável:} Apresentar ao estudante conceitos e técnicas de verificação, teste e validação de software.
	\item \textbf{Ementa:} 
	Conceitos de teste de software;
	Processo de teste e suas etapas;
	Verificação (revisão e inspeção);
	Validação (testes caixa preta e branca);
	Auditoria de Sistemas;
	Automação de teste;
	Tendências em técnicas de teste de software (\textit{Test Driven Development} e outras).
	\item \textbf{Bibliografia básica}
	\begin{enumerate}
		\item ~\cite{pressman2016engenharia}
		\item ~\cite{sommerville2011engenharia}
		\item ~\cite{rios2006teste} 
	\end{enumerate}
	\item \textbf{Bibliografia complementar}
	\begin{enumerate}
		\item ...
	\end{enumerate} 	
\end{itemize}
\nameref{tab:matriz}


%%%%%%%%%%%%%%%%%%%%%%%%%%%%%%%%%%%%%%%%%%%%%%%%%%%%%%%%%%%%%%%%%%%%%%%%%%%%%%
%%%%  Ambiente e Sociedade
%%%%%%%%%%%%%%%%%%%%%%%%%%%%%%%%%%%%%%%%%%%%%%%%%%%%%%%%%%%%%%%%%%%%%%%%%%%%%%
\newpage
\section{Ambiente e Sociedade}\label{3_educamb}
\begin{itemize}
	\item \textbf{Carga horária (hora/aula):} 27
	\item \textbf{Objetivo:} Apresentar ao estudante as relações do desenvolvimento de sistemas com o meio ambiente a a sociedade.
	\item \textbf{Ementa:} 
	A biosfera e seu equilíbrio;
	Efeitos da tecnologia sobre o equilíbrio ecológico;
	Considerações sobre poluição da água, do solo e do ar;
	Preservação dos recursos naturais: medidas de controle e tecnologia aplicada;
	Legislação ambiental;
	Avaliação de impactos ambientais de projetos tecnológicos.
	\item \textbf{Bibliografia básica}
	\begin{enumerate}
		\item 
	\end{enumerate}
	\item \textbf{Bibliografia complementar}
	\begin{enumerate}
		\item
		\item 
	\end{enumerate}
\end{itemize}
\nameref{tab:matriz}


%%%%%%%%%%%%%%%%%%%%%%%%%%%%%%%%%%%%%%%%%%%%%%%%%%%%%%%%%%%%%%%%%%%%%%%%%%%%%%
%%%%  Ambiente e Sociedade
%%%%%%%%%%%%%%%%%%%%%%%%%%%%%%%%%%%%%%%%%%%%%%%%%%%%%%%%%%%%%%%%%%%%%%%%%%%%%%
\newpage
\section{Projeto Ambiental}\label{3_projamb}
\begin{itemize}
	\item \textbf{Carga horária (hora/aula):} 54
	\item \textbf{Objetivo:} Desenvolver uma aplicação ou software relacionado ao meio ambiente utilizando o arcabouço técnico aprendido ao longo do curso.
	\item \textbf{Ementa:} 
	Prática em desenvolvimento de software;
	\item \textbf{Bibliografia básica}
	\begin{enumerate}
		\item 
	\end{enumerate}
	\item \textbf{Bibliografia complementar}
	\begin{enumerate}
		\item 	
	\end{enumerate}	
\end{itemize}
\nameref{tab:matriz}


%%%%%%%%%%%%%%%%%%%%%%%%%%%%%%%%%%%%%%%%%%%%%%%%%%%%%%%%%%%%%%%%%%%%%%%%%%%%%%
%%%%%%%%%%%%%%%%%%%%%%%%%%%%%%%%%%%%%%%%%%%%%%%%%%%%%%%%%%%%%%%%%%%%%%%%%%%%%%
%%%%  4 SEMESTRE
%%%%%%%%%%%%%%%%%%%%%%%%%%%%%%%%%%%%%%%%%%%%%%%%%%%%%%%%%%%%%%%%%%%%%%%%%%%%%%
%%%%%%%%%%%%%%%%%%%%%%%%%%%%%%%%%%%%%%%%%%%%%%%%%%%%%%%%%%%%%%%%%%%%%%%%%%%%%%


%%%%%%%%%%%%%%%%%%%%%%%%%%%%%%%%%%%%%%%%%%%%%%%%%%%%%%%%%%%%%%%%%%%%%%%%%%%%%%
%%%%  Programação para a Web I
%%%%%%%%%%%%%%%%%%%%%%%%%%%%%%%%%%%%%%%%%%%%%%%%%%%%%%%%%%%%%%%%%%%%%%%%%%%%%%
\newpage
\section{Programação para a Web I}\label{4_ppw1}
\begin{itemize}
	\item \textbf{Carga horária (hora/aula):} 54
	\item \textbf{Objetivo:} Apresentar ao estudante os fundamentos do desenvolvimento para Web utilizando linguagens de marcação e estilo, ontologias e tecnologias de mercado.
	\item \textbf{Ementa:} 
	Introdução ao desenvolvimento de aplicações Web;
	Estudo da arquitetura Web, prototipação e criação de páginas estáticas e dinâmicas com ferramentas de desenvolvimento utilizando HTML, CSS, DOM, AJAX, Javascript e XML;
	\item \textbf{Bibliografia básica}
	\begin{enumerate}
		\item 
	\end{enumerate}
	\item \textbf{Bibliografia complementar}
	\begin{enumerate}
		\item
		\item 
	\end{enumerate}
\end{itemize}
\nameref{tab:matriz}


%%%%%%%%%%%%%%%%%%%%%%%%%%%%%%%%%%%%%%%%%%%%%%%%%%%%%%%%%%%%%%%%%%%%%%%%%%%%%%
%%%%  Interação Humano-Computador
%%%%%%%%%%%%%%%%%%%%%%%%%%%%%%%%%%%%%%%%%%%%%%%%%%%%%%%%%%%%%%%%%%%%%%%%%%%%%%
\newpage
\section{Interação Humano-Computador}\label{4_ihc}
\begin{itemize}
	\item \textbf{Carga horária (hora/aula):} 27
	\item \textbf{Docente Responsável:}~\nameref{ViniciusGomes} e ~\nameref{WaldeyrMendes}
	\item \textbf{Ementa:} 
	Conceitos fundamentais em Interação Humano-Computador (IHC) e usabilidade;
	Conceitos de Engenharia Semiótica;
	Princípios de experiência do usuário e design centrado no usuário;
	Ergonomia aplicada à informática;
	Acessibilidade (diretrizes e acessibilidade para \textit{Web});
	Métodos e técnicas de avaliação de interface;
	Aspectos humanos na IHC (psicologia e fisiologia do ser humano em contato com a tecnologia).
	\item \textbf{Bibliografia básica}
	\begin{enumerate}
		\item ~\cite{Peirce1977}
	\end{enumerate}
	\item \textbf{Bibliografia complementar}
	\begin{enumerate}
		\item ~\cite{moreira1999teorias}
	\end{enumerate} 
\end{itemize}
\nameref{tab:matriz}

%%%%%%%%%%%%%%%%%%%%%%%%%%%%%%%%%%%%%%%%%%%%%%%%%%%%%%%%%%%%%%%%%%%%%%%%%%%%%%
%%%%  Administração de Serviços para a Internet
%%%%%%%%%%%%%%%%%%%%%%%%%%%%%%%%%%%%%%%%%%%%%%%%%%%%%%%%%%%%%%%%%%%%%%%%%%%%%%
\newpage
\section{Administração de Serviços para a Internet}\label{4_asi}
\begin{itemize}
	\item \textbf{Carga horária (hora/aula):} 54
	\item \textbf{Objetivo:} Proporcionar ao estudante habilidades práticas em administração e serviços para a Internet.
	\item \textbf{Ementa:} 
	Instalação e configuração de Sistemas Operacionais Servidores; Instalação e configuração de servidores: servidores web (protocolo HTTP), servidores de arquivos (protocolo SMB e FTP), servidores DHCP, servidores proxy, servidores de e-mail (protocolos SMTP, POP3, IMAP), servidores de impressão; 
	Configuração de Firewall; 
	Instalação e configuração de serviços de acesso remoto (protocolo VNC e SSH).	
	\item \textbf{Bibliografia básica}
	\begin{enumerate}
		\item 
	\end{enumerate}
	\item \textbf{Bibliografia complementar}
	\begin{enumerate}
		\item
		\item 
	\end{enumerate}
\end{itemize}
\nameref{tab:matriz}


%%%%%%%%%%%%%%%%%%%%%%%%%%%%%%%%%%%%%%%%%%%%%%%%%%%%%%%%%%%%%%%%%%%%%%%%%%%%%%
%%%%  Administração de Serviços para a Internet
%%%%%%%%%%%%%%%%%%%%%%%%%%%%%%%%%%%%%%%%%%%%%%%%%%%%%%%%%%%%%%%%%%%%%%%%%%%%%%
\newpage
\section{Introdução à Probabilidade e Estatística}\label{4_probest}
\begin{itemize}
	\item \textbf{Carga horária (hora/aula):} 54
	\item \textbf{Objetivo:} Apresentar ao estudante os fundamentos de probabilidade e estatísticas com vistas à sua aplicação em Ciência de Dados.
	\item \textbf{Ementa:} 
	Conceitos básicos de Estatística;
	Amostragem;
	Medidas de tendência central;
	Medidas de dispersão;
	Distribuição de Frequências;
	Introdução à Probabilidade;
	Distribuições teóricas de Probabilidades (Binomial, normal, t de Student, Poisson, entre outras);
	Estatística não paramétrica;
	Intervalos de Confiança;
	Testes de Hipótese;
    Correlação; 
    Regressão Linear Simples e Múltipla;
    Regressão Logística;
    Qui-Quadrado;
    Análise de Variância (ANOVA);
    Estudo de Outliers;
	Séries Temporais;
	\item \textbf{Bibliografia básica}
	\begin{enumerate}
		\item 
	\end{enumerate}
	\item \textbf{Bibliografia complementar}
	\begin{enumerate}
		\item
		\item 
	\end{enumerate}
\end{itemize}
\nameref{tab:matriz}


%%%%%%%%%%%%%%%%%%%%%%%%%%%%%%%%%%%%%%%%%%%%%%%%%%%%%%%%%%%%%%%%%%%%%%%%%%%%%%
%%%%  Relações Étnico-Raciais
%%%%%%%%%%%%%%%%%%%%%%%%%%%%%%%%%%%%%%%%%%%%%%%%%%%%%%%%%%%%%%%%%%%%%%%%%%%%%%
\newpage
\section{Relações Étnico-Raciais}\label{4_etnicoraciais}
\begin{itemize}
	\item \textbf{Carga horária (hora/aula):} 27
	\item \textbf{Objetivo:} Conhecer os conceitos de raça e etnia, mestiçagem, racismo e racialismo, fricção interétnica, preconceito e discriminação no Brasil e refletir, e debater sobre as relações interétnicas na sociedade brasileira desde a conquista até a atualidade.
	\item \textbf{Ementa:} 
	História das questões étnico-raciais no Brasil;
	Educação para as relações étnico-raciais;
	Conceitos de raça e etnia, mestiçagem, racismo e racialismo, preconceito e discriminação;
	Configurações dos conceitos de raça, etnia e cor no Brasil: abordagens acadêmicas e sociais;
	Cultura e História afro-brasileira e indígena;
	Políticas afirmativas, discriminação positiva e militância de resistência à discriminação racial e à exclusão dos negros no que tange ao acesso aos bens materiais e simbólicos produzidos no Brasil.
	\item \textbf{Bibliografia básica}
	\begin{enumerate}
		\item 
	\end{enumerate}
	\item \textbf{Bibliografia complementar}
	\begin{enumerate}
		\item
		\item 
	\end{enumerate}
\end{itemize}
\nameref{tab:matriz}

%%%%%%%%%%%%%%%%%%%%%%%%%%%%%%%%%%%%%%%%%%%%%%%%%%%%%%%%%%%%%%%%%%%%%%%%%%%%%%
%%%%  Projeto Social
%%%%%%%%%%%%%%%%%%%%%%%%%%%%%%%%%%%%%%%%%%%%%%%%%%%%%%%%%%%%%%%%%%%%%%%%%%%%%%
\newpage
\section{Projeto Social}\label{4_projsoc}
\begin{itemize}
	\item \textbf{Carga horária (hora/aula):} 54
	\item \textbf{Objetivo:} Desenvolver uma aplicação ou software com aplicação direta na sociedade local, regional ou nacional utilizando o arcabouço técnico aprendido ao longo do curso.
	\item \textbf{Ementa:} 
	Prática em desenvolvimento de software;
	\item \textbf{Bibliografia básica}
	\begin{enumerate}
		\item 
	\end{enumerate}
	\item \textbf{Bibliografia complementar}
	\begin{enumerate}
		\item 	
	\end{enumerate}	
\end{itemize}


%%%%%%%%%%%%%%%%%%%%%%%%%%%%%%%%%%%%%%%%%%%%%%%%%%%%%%%%%%%%%%%%%%%%%%%%%%%%%%
%%%%%%%%%%%%%%%%%%%%%%%%%%%%%%%%%%%%%%%%%%%%%%%%%%%%%%%%%%%%%%%%%%%%%%%%%%%%%%
%%%%  5 SEMESTRE
%%%%%%%%%%%%%%%%%%%%%%%%%%%%%%%%%%%%%%%%%%%%%%%%%%%%%%%%%%%%%%%%%%%%%%%%%%%%%%
%%%%%%%%%%%%%%%%%%%%%%%%%%%%%%%%%%%%%%%%%%%%%%%%%%%%%%%%%%%%%%%%%%%%%%%%%%%%%%


%%%%%%%%%%%%%%%%%%%%%%%%%%%%%%%%%%%%%%%%%%%%%%%%%%%%%%%%%%%%%%%%%%%%%%%%%%%%%%
%%%%  Programação para a Web II
%%%%%%%%%%%%%%%%%%%%%%%%%%%%%%%%%%%%%%%%%%%%%%%%%%%%%%%%%%%%%%%%%%%%%%%%%%%%%%
\newpage
\section{Programação para a Web II}\label{5_ppw2}
\begin{itemize}
	\item \textbf{Carga horária (hora/aula):} 54
	\item \textbf{Objetivo:} Proporcionar ao estudante prática em desenvolvimento para Web utilizando linguagens de programação, marcação e estilo, frameworks outras e tecnologias de mercado.
	\item \textbf{Ementa:} 
	Programação Web dinâmica com arquitetura cliente servidor;
	APIs;
	REST;
	Acesso a Banco de Dados;
	Frameworks Web;
	\item \textbf{Bibliografia básica}
	\begin{enumerate}
		\item 
	\end{enumerate}
	\item \textbf{Bibliografia complementar}
	\begin{enumerate}
		\item
		\item 
	\end{enumerate}
\end{itemize}
\nameref{tab:matriz}


%%%%%%%%%%%%%%%%%%%%%%%%%%%%%%%%%%%%%%%%%%%%%%%%%%%%%%%%%%%%%%%%%%%%%%%%%%%%%%
%%%%  Programação para a Web II
%%%%%%%%%%%%%%%%%%%%%%%%%%%%%%%%%%%%%%%%%%%%%%%%%%%%%%%%%%%%%%%%%%%%%%%%%%%%%%
\newpage
\section{Metodologia da Pesquisa Científica}\label{5_metodologia}
\begin{itemize}
	\item \textbf{Carga horária (hora/aula):} 54
	\item \textbf{Objetivo:} Proporcionar ao estudante conhecer conceitos e técnicas de metodologia científica com vistas ao desenvolvimento de projetos acadêmicos incluindo seu Trabalho de Conclusão de Curso.
	\item \textbf{Ementa:} 
	Conhecimento científico;
	Métodos de pesquisa; 
	Revisão bibliográfica;
	Pesquisa qualitativa;
	Pesquisa quantitativa; 
	Redação técnica;
	Trabalhos acadêmicos;
	Portais e bases de conhecimento;
	Bibliometria;
	Construção do pré-projeto de trabalho de conclusão de curso (TCC).
	\item \textbf{Bibliografia básica}
	\begin{enumerate}
		\item \cite{Andrade2005introduccao}
		\item \cite{gil2002elaborar}
		\item \cite{wazlawick2017metodologia}
	\end{enumerate}
	\item \textbf{Bibliografia complementar}
	\begin{enumerate}
		\item \cite{koche2016fundamentos}
		\item \cite{aquino2017escrever}~\footnote{Não tem na biblioteca do câmpus Formosa.}
	\end{enumerate} 	
\end{itemize}
\nameref{tab:matriz}


%%%%%%%%%%%%%%%%%%%%%%%%%%%%%%%%%%%%%%%%%%%%%%%%%%%%%%%%%%%%%%%%%%%%%%%%%%%%%%
%%%% Libras
%%%%%%%%%%%%%%%%%%%%%%%%%%%%%%%%%%%%%%%%%%%%%%%%%%%%%%%%%%%%%%%%%%%%%%%%%%%%%%
\newpage
\section{LIBRAS}\label{5_libras}
\begin{itemize}
	\item \textbf{Carga horária (hora/aula):} 27
	\item \textbf{Objetivo:} Instrumentalizar ao estudante o estabelecimento de uma comunicação funcional com pessoas surdas;
	\item \textbf{Ementa:} 
	Aspectos clínicos, educacionais e sócio-antropológicos da surdez; 
	A Língua de Sinais Brasileira -	Libras: características básicas da fonologia;
	Noções básicas de léxico, de morfologia e de sintaxe com apoio de recursos audio-visuais; 
	Noções de variação; 
	Praticar Libras: desenvolver a expressão visual-espacial para a sociedade e para o ensino de Biologia;
	\item \textbf{Bibliografia Básica}
	\begin{enumerate}
		\item \cite{cartilha2012}
		\item 
		\item 
	\end{enumerate}
	\item \textbf{Bibliografia complementar}
	\begin{enumerate}
		\item 
		\item
	\end{enumerate} 	
\end{itemize}
\nameref{tab:matriz}


%%%%%%%%%%%%%%%%%%%%%%%%%%%%%%%%%%%%%%%%%%%%%%%%%%%%%%%%%%%%%%%%%%%%%%%%%%%%%%
%%%% Ética e Legislação
%%%%%%%%%%%%%%%%%%%%%%%%%%%%%%%%%%%%%%%%%%%%%%%%%%%%%%%%%%%%%%%%%%%%%%%%%%%%%%
\newpage
\section{Ética e Legislação}\label{5_etica}
\begin{itemize}
	\item \textbf{Carga horária (hora/aula):} 27
	\item \textbf{Objetivo:} Apresentar ao estudante os principais marcos regulatórios e de conduta relacionado à profissão de analista e/ou desenvolvedor de sistemas.
	\item \textbf{Ementa:} 
	Lei Geral de Proteção de Dados;
	\item \textbf{Bibliografia Básica}
	\begin{enumerate}
		\item \cite{cartilha2012}
		\item 
		\item 
	\end{enumerate}
	\item \textbf{Bibliografia complementar}
	\begin{enumerate}
		\item 
		\item
	\end{enumerate} 	
\end{itemize}
\nameref{tab:matriz}

%%%%%%%%%%%%%%%%%%%%%%%%%%%%%%%%%%%%%%%%%%%%%%%%%%%%%%%%%%%%%%%%%%%%%%%%%%%%%%
%%%%  Disciplina Eletiva
%%%%%%%%%%%%%%%%%%%%%%%%%%%%%%%%%%%%%%%%%%%%%%%%%%%%%%%%%%%%%%%%%%%%%%%%%%%%%%
\newpage
\section{Disciplina Eletiva}\label{5_opt}
\begin{itemize}
	\item \textbf{Carga horária (hora/aula):} 54
	\item \textbf{Objetivo:} Ofertar ao estudante a possibilidade de conhecer conceitos, técnicas e tecnologias emergentes.
	\item \textbf{Ementa:} 
	Ementa variável;
	\item \textbf{Bibliografia básica}
	\begin{enumerate}
		\item 
	\end{enumerate}
	\item \textbf{Bibliografia complementar}
	\begin{enumerate}
		\item 	
	\end{enumerate}	
\end{itemize}


%%%%%%%%%%%%%%%%%%%%%%%%%%%%%%%%%%%%%%%%%%%%%%%%%%%%%%%%%%%%%%%%%%%%%%%%%%%%%%
%%%%  Laboratório de Análise e Desenvolvimento de Software
%%%%%%%%%%%%%%%%%%%%%%%%%%%%%%%%%%%%%%%%%%%%%%%%%%%%%%%%%%%%%%%%%%%%%%%%%%%%%%
\newpage
\section{Laboratório de Análise e Desenvolvimento de Software}\label{5_lab}
\begin{itemize}
	\item \textbf{Carga horária (hora/aula):} 54
	\item \textbf{Objetivo:} Desenvolver uma aplicação ou software de tema específico como pré-projeto para o Trabalho de Conclusão de Curso.
	\item \textbf{Ementa:} 
	Prática em desenvolvimento de software;
	\item \textbf{Bibliografia básica}
	\begin{enumerate}
		\item 
	\end{enumerate}
	\item \textbf{Bibliografia complementar}
	\begin{enumerate}
		\item 	
	\end{enumerate}	
\end{itemize}


%%%%%%%%%%%%%%%%%%%%%%%%%%%%%%%%%%%%%%%%%%%%%%%%%%%%%%%%%%%%%%%%%%%%%%%%%%%%%%
%%%%%%%%%%%%%%%%%%%%%%%%%%%%%%%%%%%%%%%%%%%%%%%%%%%%%%%%%%%%%%%%%%%%%%%%%%%%%%
%%%%  6 SEMESTRE
%%%%%%%%%%%%%%%%%%%%%%%%%%%%%%%%%%%%%%%%%%%%%%%%%%%%%%%%%%%%%%%%%%%%%%%%%%%%%%
%%%%%%%%%%%%%%%%%%%%%%%%%%%%%%%%%%%%%%%%%%%%%%%%%%%%%%%%%%%%%%%%%%%%%%%%%%%%%%

%%%%%%%%%%%%%%%%%%%%%%%%%%%%%%%%%%%%%%%%%%%%%%%%%%%%%%%%%%%%%%%%%%%%%%%%%%%%%%
%%%% Fundamentos de Inteligência Artificial
%%%%%%%%%%%%%%%%%%%%%%%%%%%%%%%%%%%%%%%%%%%%%%%%%%%%%%%%%%%%%%%%%%%%%%%%%%%%%%
\newpage
\section{Fundamentos de Inteligência Artificial}\label{6_ia}
\begin{itemize}
	\item \textbf{Carga horária (hora/aula):} 54
	\item \textbf{Objetivo:} Proporcionar ao estudante conhecer e aplicar técnicas de intelência artificial na solução de problemas e no desenvolvimento de software.
	\item \textbf{Ementa:} 
	Conceitos de Inteligência Artificial (IA);
	Algoritmos de busca e otimização;
	Algoritmos genéticos;
	Sistemas baseados em conhecimento (Sistemas especialistas e lógica difusa);
	Machine Learning e seus algoritmos, redes neurais e deep learning, reinforcement learning;
	Processamento de linguagem natural; 
	Tendências em IA;
	\item \textbf{Bibliografia Básica}
	\begin{enumerate}
		\item \cite{barbetta2004estatistica}
		\item 
		\item 
	\end{enumerate}
	\item \textbf{Bibliografia complementar}
	\begin{enumerate}
		\item \cite{farberestatistica}
		\item
	\end{enumerate} 	
\end{itemize}
\nameref{tab:matriz}

%%%%%%%%%%%%%%%%%%%%%%%%%%%%%%%%%%%%%%%%%%%%%%%%%%%%%%%%%%%%%%%%%%%%%%%%%%%%%%
%%%%  Segurança da Informação
%%%%%%%%%%%%%%%%%%%%%%%%%%%%%%%%%%%%%%%%%%%%%%%%%%%%%%%%%%%%%%%%%%%%%%%%%%%%%%
\newpage
\section{Segurança da Informação}\label{6_seginfo}
\begin{itemize}
	\item \textbf{Carga horária (hora/aula):} 27
	\item \textbf{Objetivo:} Apresentar ao estudante os fundamentos de segurança da informação como apoio no desenvolvimento para sistemas seguros.
	\item \textbf{Ementa:} 
	Visão geral sobre segurança de sistemas;
	Políticas de segurança; 
	Privacidade na era digital;
	Análise de riscos em sistemas de informação;
	Vírus;
	Criptografia; 
	Acesso não autorizado; 
	Ataques;
	Firewall;
	Mecanismos de criptografia (simétrica e assimétrica);
	Assinatura digital e certificados digitais;
	\item \textbf{Bibliografia Básica}
	\begin{enumerate}
		\item \cite{cartilha2012}
		\item 
		\item 
	\end{enumerate}
	\item \textbf{Bibliografia complementar}
	\begin{enumerate}
		\item 
		\item
	\end{enumerate} 	
\end{itemize}
\nameref{tab:matriz}


%%%%%%%%%%%%%%%%%%%%%%%%%%%%%%%%%%%%%%%%%%%%%%%%%%%%%%%%%%%%%%%%%%%%%%%%%%%%%%
%%%%  Empreendedorismo Digital
%%%%%%%%%%%%%%%%%%%%%%%%%%%%%%%%%%%%%%%%%%%%%%%%%%%%%%%%%%%%%%%%%%%%%%%%%%%%%%
\newpage
\section{Empreendedorismo Digital}\label{6_empdig}
\begin{itemize}
	\item \textbf{Carga horária (hora/aula):} 27
	\item \textbf{Objetivo:} Apresentar ao estudante coceitos e técnicas de empreendedorismo e startups;
	\item \textbf{Ementa:} 
	Conceitos iniciais sobre empreendedorismo;
	Startups e o Processo de Desenvolvimento de Clientes;
	Design Thinking;
	Canvas (Business Model Canvas, Project Model Canvas, Propostion Value Canvas, Lean Canvas, etc.);
    Diferentes tipos de pitchs (Elevator Pitch, Pitch Desk, etc.);
    Marketing Digital para desenvolvedores de software (Growth, Teste A/B, Marketing de redes sociais, etc.);
    
	\item \textbf{Bibliografia Básica}
	\begin{enumerate}
		\item 
		\item 
		\item 
	\end{enumerate}
	\item \textbf{Bibliografia complementar}
	\begin{enumerate}
		\item 
		\item
	\end{enumerate} 	
\end{itemize}
\nameref{tab:matriz}


%%%%%%%%%%%%%%%%%%%%%%%%%%%%%%%%%%%%%%%%%%%%%%%%%%%%%%%%%%%%%%%%%%%%%%%%%%%%%%
%%%% Fundamentos de Ciência de Dados
%%%%%%%%%%%%%%%%%%%%%%%%%%%%%%%%%%%%%%%%%%%%%%%%%%%%%%%%%%%%%%%%%%%%%%%%%%%%%%
\newpage
\section{Fundamentos de Ciência de Dados}\label{6_datascience}
\begin{itemize}
	\item \textbf{Carga horária (hora/aula):} 54
	\item \textbf{Objetivo:} Proporcionar ao estudante conhecer e aplicar técnicas para análise de dados, incluindo Big Data.
	\item \textbf{Ementa:} 
	Introdução à Ciência de Dados;
	Conceitos estatísticos para análise de dados;
	Métodos de Aprendizagem Automática: Supervisionados (Regressão, Classificação e outros) e Não-supervisionados (Agrupamentos e outros);
	Introdução às linguagens Python e R;
	\item \textbf{Bibliografia Básica}
	\begin{enumerate}
		\item \cite{barbetta2004estatistica}
		\item 
		\item 
	\end{enumerate}
	\item \textbf{Bibliografia complementar}
	\begin{enumerate}
		\item \cite{farberestatistica}
		\item
	\end{enumerate} 	
\end{itemize}
\nameref{tab:matriz}

%%%%%%%%%%%%%%%%%%%%%%%%%%%%%%%%%%%%%%%%%%%%%%%%%%%%%%%%%%%%%%%%%%%%%%%%%%%%%%
%%%%  Visualização de Dados e \textit{Storytelling}
%%%%%%%%%%%%%%%%%%%%%%%%%%%%%%%%%%%%%%%%%%%%%%%%%%%%%%%%%%%%%%%%%%%%%%%%%%%%%%
\newpage
\section{Visualização de Dados e \textit{Storytelling}}\label{6_visstory}
\begin{itemize}
	\item \textbf{Carga horária (hora/aula):} 54
	\item \textbf{Objetivo:} Apresentar ao estudante conceitos e técnicas de visualização de dados.
	\item \textbf{Ementa:} 
	Introdução à Visualização de Dados;
	Métodos de Visualização;
	Organização Visual;
	Dashboard Desgin;
	Data Storytelling;
	Linguagens, ferramentas e bibliotecas de Visualização de Dados (R, Python, Tableau, Qlik Sense, etc.)
	\item \textbf{Bibliografia Básica}
	\begin{enumerate}
		\item 
		\item 
		\item 
	\end{enumerate}
	\item \textbf{Bibliografia complementar}
	\begin{enumerate}
		\item 
		\item
	\end{enumerate} 	
\end{itemize}
\nameref{tab:matriz}]

%%%%%%%%%%%%%%%%%%%%%%%%%%%%%%%%%%%%%%%%%%%%%%%%%%%%%%%%%%%%%%%%%%%%%%%%%%%%%%
%%%%  Governança e Gerência de Projetos de Tecnologia da Informação
%%%%%%%%%%%%%%%%%%%%%%%%%%%%%%%%%%%%%%%%%%%%%%%%%%%%%%%%%%%%%%%%%%%%%%%%%%%%%%
\newpage
\section{Governança e Gerência de Projetos de Tecnologia da Informação}\label{6_govproj}
\begin{itemize}
	\item \textbf{Carga horária (hora/aula):} 54
	\item \textbf{Objetivo:} Apresentar ao estudante conceitos e frameworks de melhore spráticas em gerência e governança de projetos de Tecnologia da Informação.
	\item \textbf{Ementa:} 
	Introdução ao Projeto (Projeto x Operação, Estruturas Organizacionais, Papel do Gerente de Projetos);
	Ciclo de Vida e Áreas de Conhecimento do PMBOK;
	Gerenciamento de Projetos com métodos ágeis;
	COBIT;
	ITIL;
	\item \textbf{Bibliografia Básica}
	\begin{enumerate}
		\item 
		\item 
		\item 
	\end{enumerate}
	\item \textbf{Bibliografia complementar}
	\begin{enumerate}
		\item 
		\item
	\end{enumerate} 	
\end{itemize}
\nameref{tab:matriz}

%----------------------------------------------------------------------------------------
%	CHAPTER X
%----------------------------------------------------------------------------------------
%------------------------------------------------
\chapterimage{05.jpg} % Chapter heading image
\chapter{Estrutura Física}\label{estrutura}
\vspace{6em}
\begin{flushright}
	\textit{\textcolor{white}{Um bonita citação...}}
\end{flushright}
\vspace{12em}

\section{Laboratórios de Informática}
Dois laboratórios de informática com capacidade para até 30 estudantes, com acesso à Internet, computadores com sistema operacional Linux, softwares diversos.

\section{Laboratório de Fisiologia Vegetal}
Equipado com: estufa de secagem, 3 estereoscópios, 3 microscópicos, geladeira, bancadas, 28 cadeiras, quadro e acervo didático (frutos, sementes e folhas herborizadas). 

\section{Laboratório de Bioqímica}
Equipado com: Balanças analítica e semi-analítica, chapas de aquecimento (com agitação magnética), analisador bioquímico, capela de fluxo laminar, agitadores de tubo de ensaio, banho-maria, bomba de vácuo, autoclave, estufas, destilador e deionizador de água e outros.

\section{Laboratório de Anatomia e Zoologia}
Equipado com: Bonecos anatômicos (de abdome) completos, conjuntos anatômicos artificiais de sistemas reprodutores femininos e masculinos, esqueletos completos (artificiais), amostras de animais (do cerrado e de outros biomas) conservados em frascos para visualização, animais empalhados, algumas peças anatômicas naturais de animais, lupas, microscópios, material para coleta de animais e saídas de campo, materiais e reagentes para o empalhamento de animais e outros.

\section{Laboratório de Microscopia e Microbiologia}
Equipado com:  25 microscópios e material para produção de lâminas (lâminas de corte, lâminas e lamínulas de vidro, corantes, fixadores, etc); Lupas, coleções de laminários e outros.

\section{Laboratório de Físico-Química}
Equipado com: pHmetros, destilador, capela de exaustão, estufa, banho-maria, balanças analítica e semi-analítica, deionizador, reator, aparelho de ponto de fusão,  e outros.

\section{Laboratório de Águas Residuais}
Equipado com: Condutivímetros, muflas, banho - maria, bomba de vácuo, analisador de oxigênio dissolvido, turbdímetro, estufa, balança, phmetro, destilador e outros.

\section{Laboratório de Ensino}
Espaço acadêmico voltado ao desenvolvimento e disseminação de tecnologias educacionais voltadas ao ensino de Ciências e Biologia.  Equipado com: acervo didático constituído por jogos, maquetes e representações físicas de organismos e processos biológicos.

\section{Laboratório de Física e Matemática}
O Laboratório de Física possui diversos equipamentos que contribui para o desenvolvimento das atividades experimentais nas áreas de mecânica, óptica, hidrostática, termologia e eletricidade.

\section{Biblioteca}
Biblioteca equipada com áreas de estudo individual e coletivo, 6 computadores com acesso ao portal de periódicos e acervo cerca de 7 mil exemplares, entre livros, livros em braile, cds, dvds e mapas;

\section{Teatro}
Teatro equipado com som e iluminação específica e acomodações para 320 pessoas sentadas;

\section{Outros Espaços}
3 salas para estudos coletivos e reuniões equipadas com mesas, cadeiras e televisor.


%----------------------------------------------------------------------------------------
%	CHAPTER X
%----------------------------------------------------------------------------------------
%------------------------------------------------
\chapterimage{06.jpg} % Chapter heading image
\chapter{Corpo Docente}\label{docentes}
\vspace{6em}
\begin{flushright}
	\textit{\textcolor{white}{Um bonita citação...}}
\end{flushright}
\vspace{12em}


\section{Vinícius Gomes}\label{ViniciusGomes}
\begin{itemize}
	\item Formação Básica: 
	\item Titulação Máxima: 
	\item Regime de Trabalho: Dedicação Exclusiva
	\item \includegraphics[scale=.03]{Pictures/lattes}~\href{http://lattes.cnpq.br/2391349697609978}{Lattes: http://lattes.cnpq.br/2391349697609978}
	\item \includegraphics[scale=.15]{Pictures/orcid}~\href{https://orcid.org/0000-0002-8660-6331}{ORCID: https://orcid.org/0000-0002-8660-6331}
\end{itemize}


\section{Waldeyr Mendes Cordeiro da Silva}\label{WaldeyrMendes}
\begin{itemize}
	\item Formação Básica: Sistemas de Informação e Ciências Biológicas 
	\item Titulação Máxima: Doutor em Ciências Biológicas (Bioinformática)
	\item Regime de Trabalho: Dedicação Exclusiva
	\item \includegraphics[scale=.03]{Pictures/lattes}~\href{http://lattes.cnpq.br/2391349697609978}{Lattes: http://lattes.cnpq.br/2391349697609978}
	\item \includegraphics[scale=.15]{Pictures/orcid}~\href{https://orcid.org/0000-0002-8660-6331}{ORCID: https://orcid.org/0000-0002-8660-6331}
\end{itemize}




% ----------------------------------------------------------------------------------------
% 	BIBLIOGRAPHY
% ----------------------------------------------------------------------------------------
%----------------------------------------------------------------------------------------
%	CHAPTER X
%----------------------------------------------------------------------------------------
%------------------------------------------------
\chapterimage{07.jpg} % Chapter heading image
%\chapter*{Referências Bibliográficas}
%\bibliography{bibliography}
%\renewcommand\bibname{Referências Bibliográficas}

\chapter*{Referências Bibliográficas}\label{referencias}
\vspace{6em}
\begin{flushright}
	\textit{\textcolor{white}{Um bonita citação...}}
\end{flushright}
\vspace{12em}
%\addcontentsline{toc}{chapter}{\textcolor{verde}{Bibliography}}
%\section{Books}
%\addcontentsline{toc}{section}{Books}
%\printbibliography[heading=bibempty,type=book]
%\section{Articles}
%\addcontentsline{toc}{section}{Articles}
%\printbibliography[heading=bibempty,type=article]
\printbibliography[heading=bibempty]


%----------------------------------------------------------------------------------------

%----------------------------------------------------------------------------------------
%	INDEX
%----------------------------------------------------------------------------------------
%
%\cleardoublepage
%\phantomsection
%\setlength{\columnsep}{0.75cm}
%\addcontentsline{toc}{chapter}{\textcolor{verde}{Index}}
%\printindex


%----------------------------------------------------------------------------------------
%	CHAPTER X
%----------------------------------------------------------------------------------------
%------------------------------------------------
\chapterimage{08.jpg} % Chapter heading image
\chapter{Anexos}\label{conheca}
\vspace{6em}
\begin{flushright}
	\textit{\textcolor{white}{}}
\end{flushright}
\vspace{12em}

\newpage
\section{Compatibilidade com a Matriz Curricular Anterior}
\indent
%	\caption{Tabela de compatibilidade de disciplinas entre a matriz curricular antiga e a atualizada.}

Com a atualização da Matriz Curricular do Curso Superior de Tecnologia em Análise e Desenvolvimento de Sistemas, estudantes matriculados na matriz anterior poderão migrar para a nova matriz respeitadas as equivalências definidas na Tabela~\ref{tab:compatibilidade}.


% Please add the following required packages to your document preamble:
% \usepackage{booktabs}
% \usepackage{graphicx}
\begin{table}[]
	\centering
	\caption{Tabela de equivalência de disciplinas entre a Matriz Curricular de 2014 e a Matriz Vigente.}
	\label{tab:compatibilidade}
	\resizebox{\textwidth}{!}{%
		\begin{tabular}{@{}l|l@{}}
			\toprule
			\textbf{Disciplina}             & \textbf{Equivalência na Matriz 2014}                                     \\ \midrule
			\nameref{1_algoritmos}          & Algoritmos                                                                \\
			\nameref{1_fundcomp}            & Fundamentos da Computação                                                 \\
			\nameref{1_engsof}              & Engenharia de Software                                                    \\
			\nameref{1_matematica}          & Matemática Elementar                                                      \\
			\nameref{1_leitprodtextos}      & -                                                                         \\
			\nameref{1_inglacad}            & Inglês Instrumental + Tópicos avançados I                                 \\ \midrule
			\nameref{2_sistop}              & Sistemas Operacionais                                                     \\
			\nameref{2_bancodados}          & Banco de Dados I                                                          \\
			\nameref{2_arqsoft}             & Arquitetura e Projeto de Software                                         \\
			\nameref{2_estruturadedados}    & Estrutura de Dados I                                                      \\
			\nameref{2_algebra}             & Cálculo Diferencial e Integral                                            \\ \midrule
			\nameref{3_poo}                 & Programação Orientada a Objetos                                           \\
			\nameref{3_nosql}               & Fundamentos de Sistemas de Informação                                     \\			
			\nameref{3_testsoft}            & Qualidade de Software                                                     \\
			\nameref{3_engreq}              & Engenharia de Requisitos                                                  \\
			\nameref{3_redescomp}           & Redes de Computadores                                                     \\
			\nameref{3_educamb}             & Educação Ambiental                                                        \\
			\nameref{3_projamb}             & Lógica Computacional                                                      \\ \midrule
			\nameref{4_ppw1}                & Programação para Web I                                                    \\
			\nameref{4_ihc}                 & Interface Homem Computador                                                \\
			\nameref{4_asi}                 & Administração de Serviços para Internet                                   \\
			\nameref{4_probest}             & Introdução à Probabilidade e Estatística + Análise Orientada a Objetos    \\
			\nameref{4_etnicoraciais}       & Relações étnico-raciais, História e Cultura Afro-Brasileira e Indígena    \\
			\nameref{4_projsoc}             & -                                                                         \\ \midrule			
			\nameref{5_ppw2}                & Programação para Web II                                                   \\
			\nameref{5_metodologia}         & Metodologia da Pesquisa Científica + Sociologia do Trabalho               \\
			\nameref{5_libras}              & Libras                                                                    \\
			\nameref{5_etica}               & Ética e Legislação Aplicada à Informática                                 \\
			\nameref{5_opt}                 & Sistemas Distribuídos                                                     \\
			\nameref{5_lab}                 & Métodos e Técnicas de Programação                                         \\ \midrule	
			\nameref{6_ia}                  & Estrutura de Dados II                                                     \\
			\nameref{6_seginfo}             & Segurança e Auditoria de Sistemas                                         \\
			\nameref{6_empdig}              & Tópicos Avançados II                                                      \\
			\nameref{6_datascience}         & Banco de Dados II                                                         \\
			\nameref{6_visstory}            & Computação Gráfica e Sistemas Multimídia                                  \\
			\nameref{6_govproj}             & Gerência de Projetos                                                       \\ \bottomrule
		\end{tabular}%
	}
\end{table}

\end{document}
